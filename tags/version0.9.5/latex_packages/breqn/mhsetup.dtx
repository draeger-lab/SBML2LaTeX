% \iffalse meta-comment
%
% Copyright (C) 2002-2007 by Morten Hoegholm <mh.ctan@gmail.com>
%
% This work may be distributed and/or modified under the
% conditions of the LaTeX Project Public License, either
% version 1.3 of this license or (at your option) any later
% version. The latest version of this license is in
%    http://www.latex-project.org/lppl.txt
% and version 1.3 or later is part of all distributions of
% LaTeX version 2005/12/01 or later.
%
% This work has the LPPL maintenance status "maintained".
%
% This Current Maintainer of this work is Morten Hoegholm.
%
% This work consists of the main source file mhsetup.dtx
% and the derived files
%    mhsetup.sty, mhsetup.pdf, mhsetup.ins, mhsetup.drv.
%
% Distribution:
%    CTAN:macros/latex/contrib/mh/mhsetup.dtx
%    CTAN:macros/latex/contrib/mh/mhsetup.pdf
%
% Unpacking:
%    (a) If mhsetup.ins is present:
%           tex mhsetup.ins
%    (b) Without mhsetup.ins:
%           tex mhsetup.dtx
%    (c) If you insist on using LaTeX
%           latex \let\install=y% \iffalse meta-comment
%
% Copyright (C) 2002-2007 by Morten Hoegholm <mh.ctan@gmail.com>
%
% This work may be distributed and/or modified under the
% conditions of the LaTeX Project Public License, either
% version 1.3 of this license or (at your option) any later
% version. The latest version of this license is in
%    http://www.latex-project.org/lppl.txt
% and version 1.3 or later is part of all distributions of
% LaTeX version 2005/12/01 or later.
%
% This work has the LPPL maintenance status "maintained".
%
% This Current Maintainer of this work is Morten Hoegholm.
%
% This work consists of the main source file mhsetup.dtx
% and the derived files
%    mhsetup.sty, mhsetup.pdf, mhsetup.ins, mhsetup.drv.
%
% Distribution:
%    CTAN:macros/latex/contrib/mh/mhsetup.dtx
%    CTAN:macros/latex/contrib/mh/mhsetup.pdf
%
% Unpacking:
%    (a) If mhsetup.ins is present:
%           tex mhsetup.ins
%    (b) Without mhsetup.ins:
%           tex mhsetup.dtx
%    (c) If you insist on using LaTeX
%           latex \let\install=y% \iffalse meta-comment
%
% Copyright (C) 2002-2007 by Morten Hoegholm <mh.ctan@gmail.com>
%
% This work may be distributed and/or modified under the
% conditions of the LaTeX Project Public License, either
% version 1.3 of this license or (at your option) any later
% version. The latest version of this license is in
%    http://www.latex-project.org/lppl.txt
% and version 1.3 or later is part of all distributions of
% LaTeX version 2005/12/01 or later.
%
% This work has the LPPL maintenance status "maintained".
%
% This Current Maintainer of this work is Morten Hoegholm.
%
% This work consists of the main source file mhsetup.dtx
% and the derived files
%    mhsetup.sty, mhsetup.pdf, mhsetup.ins, mhsetup.drv.
%
% Distribution:
%    CTAN:macros/latex/contrib/mh/mhsetup.dtx
%    CTAN:macros/latex/contrib/mh/mhsetup.pdf
%
% Unpacking:
%    (a) If mhsetup.ins is present:
%           tex mhsetup.ins
%    (b) Without mhsetup.ins:
%           tex mhsetup.dtx
%    (c) If you insist on using LaTeX
%           latex \let\install=y% \iffalse meta-comment
%
% Copyright (C) 2002-2007 by Morten Hoegholm <mh.ctan@gmail.com>
%
% This work may be distributed and/or modified under the
% conditions of the LaTeX Project Public License, either
% version 1.3 of this license or (at your option) any later
% version. The latest version of this license is in
%    http://www.latex-project.org/lppl.txt
% and version 1.3 or later is part of all distributions of
% LaTeX version 2005/12/01 or later.
%
% This work has the LPPL maintenance status "maintained".
%
% This Current Maintainer of this work is Morten Hoegholm.
%
% This work consists of the main source file mhsetup.dtx
% and the derived files
%    mhsetup.sty, mhsetup.pdf, mhsetup.ins, mhsetup.drv.
%
% Distribution:
%    CTAN:macros/latex/contrib/mh/mhsetup.dtx
%    CTAN:macros/latex/contrib/mh/mhsetup.pdf
%
% Unpacking:
%    (a) If mhsetup.ins is present:
%           tex mhsetup.ins
%    (b) Without mhsetup.ins:
%           tex mhsetup.dtx
%    (c) If you insist on using LaTeX
%           latex \let\install=y\input{mhsetup.dtx}
%        (quote the arguments according to the demands of your shell)
%
% Documentation:
%    (a) If mhsetup.drv is present:
%           latex mhsetup.drv
%    (b) Without mhsetup.drv:
%           latex mhsetup.dtx; ...
%    The class ltxdoc loads the configuration file ltxdoc.cfg
%    if available. Here you can specify further options, e.g.
%    use A4 as paper format:
%       \PassOptionsToClass{a4paper}{article}
%
%    Programm calls to get the documentation (example):
%       pdflatex mhsetup.dtx
%       makeindex -s gind.ist mhsetup.idx
%       pdflatex mhsetup.dtx
%       makeindex -s gind.ist mhsetup.idx
%       pdflatex mhsetup.dtx
%
% Installation:
%    TDS:tex/latex/mh/mhsetup.sty
%    TDS:doc/latex/mh/mhsetup.pdf
%    TDS:source/latex/mh/mhsetup.dtx
%
%<*ignore>
\begingroup
  \def\x{LaTeX2e}
\expandafter\endgroup
\ifcase 0\ifx\install y1\fi\expandafter
         \ifx\csname processbatchFile\endcsname\relax\else1\fi
         \ifx\fmtname\x\else 1\fi\relax
\else\csname fi\endcsname
%</ignore>
%<*install>
\input docstrip.tex
\Msg{************************************************************************}
\Msg{* Installation}
\Msg{* Package: mhsetup 2007/12/03 v1.2 mhsetup (MH)}
\Msg{************************************************************************}

\keepsilent
\askforoverwritefalse

\preamble

This is a generated file.

Copyright (C) 2002-2007 by Morten Hoegholm <mh.ctan@gmail.com>

This work may be distributed and/or modified under the
conditions of the LaTeX Project Public License, either
version 1.3 of this license or (at your option) any later
version. The latest version of this license is in
   http://www.latex-project.org/lppl.txt
and version 1.3 or later is part of all distributions of
LaTeX version 2005/12/01 or later.

This work has the LPPL maintenance status "maintained".

This Current Maintainer of this work is Morten Hoegholm.

This work consists of the main source file mhsetup.dtx
and the derived files
   mhsetup.sty, mhsetup.pdf, mhsetup.ins, mhsetup.drv.

\endpreamble

\generate{%
  \file{mhsetup.ins}{\from{mhsetup.dtx}{install}}%
  \file{mhsetup.drv}{\from{mhsetup.dtx}{driver}}%
  \usedir{tex/latex/mh}%
  \file{mhsetup.sty}{\from{mhsetup.dtx}{package}}%
}

\obeyspaces
\Msg{************************************************************************}
\Msg{*}
\Msg{* To finish the installation you have to move the following}
\Msg{* file into a directory searched by TeX:}
\Msg{*}
\Msg{*     mhsetup.sty}
\Msg{*}
\Msg{* To produce the documentation run the file `mhsetup.drv'}
\Msg{* through LaTeX.}
\Msg{*}
\Msg{* Happy TeXing!}
\Msg{*}
\Msg{************************************************************************}

\endbatchfile
%</install>
%<*ignore>
\fi
%</ignore>
%<*driver>
\NeedsTeXFormat{LaTeX2e}
\ProvidesFile{mhsetup.drv}%
  [2007/12/03 v1.2 programming setup (MH)]
\documentclass{ltxdoc}
\usepackage{mhsetup}

\addtolength\oddsidemargin{30pt}
\providecommand*\pkg[1]{\textsf{#1}}
\providecommand*\env[1]{\texttt{#1}}
\providecommand*\opt[1]{\texttt{#1}}
\def\MakePrivateLetters{%
  \makeatletter\catcode`\_=11\relax \catcode`\:=11\relax
}



\def\MHmarg#1{{\ttfamily\char`\{}\meta{#1}{\ttfamily\char`\}}\endgroup}
\DeclareRobustCommand\marg{\begingroup\MakePrivateLetters\MHmarg}%
\def\MHarg#1{{\ttfamily\char`\{#1\ttfamily\char`\}}\endgroup}
\DeclareRobustCommand\arg{\begingroup\MakePrivateLetters\MHarg}%
\def\MHcs#1{\texttt{\char`\\#1}\endgroup}
\DeclareRobustCommand\cs{\begingroup\MakePrivateLetters\MHcs}
\makeatletter
\providecommand*\eTeX{$\m@th\varepsilon$-\TeX}
\makeatother
\begin{document}
  \DocInput{mhsetup.dtx}
\end{document}
%</driver>
%  \fi
%
%  \CheckSum{297}
%
%  \GetFileInfo{mhsetup.drv}
%
%  \title{The \pkg{mhsetup} package\thanks{This package has version
%  number \fileversion, last revised on \filedate.}}
%  \author{Morten H\o gholm}
%  \date{\filedate}
%  \maketitle
%
%  \begin{abstract}
%    The \pkg{mhsetup} package provides tools for a \LaTeX{}
%    programming environment similar to the one described in
%    \texttt{expl3} on \textsc{ctan} although not as extensive. It is
%    a required part of both the \pkg{mathtools} and \pkg{empheq}
%    packages.
%  \end{abstract}
%
%  The description below was made before the extensive changes made to
%  the expl3 code available from the LaTeX Project website.
%
%  \section{The new internal syntax}
%
%  
%  The \LaTeX3 package \pkg{ldcsetup} defines the command
%  \cs{InternalSyntaxOn} which makes |_| and |:| letters and then
%  automatically restores the category codes at the end of the
%  package. This usually works fine but when you try to load
%  \pkg{amstext} you will experience that \TeX{} goes into an
%  infinite loop. Packages containing code like
%  |\@for\@tempa:=\@tempb\do{...}| will not work correctly either,
%  thus we provide an alternative version here with the pair of
%  commands
%  \DescribeMacro{\MHInternalSyntaxOn}\cs{MHInternalSyntaxOn} and
%  \DescribeMacro{\MHInternalSyntaxOff}\cs{MHInternalSyntaxOff}. They
%  are to be used only as a pair, because \cs{MHInternalSyntaxOn}
%  defines \cs{MHInternalSyntaxOff} so that it restores the category
%  codes correctly.
%
%
%  \section{Handling optional arguments}
%
%  The standard behavior of scanning for optional arguments in
%  \LaTeX{} allows any number of spaces preceding the optional
%  argument and that is not always good in math. For that reason
%  \pkg{amsmath} makes sure that commands like |\\| disallows spaces
%  before the optional argument but at the same time it fails to
%  provide ``safe'' environments. What would you expect from the
%  following input?
%  \begin{verbatim}
%  \[
%    \begin{gathered}
%      [v] = 100 \\
%      [t] = 200
%    \end{gathered}
%  \]
%  \end{verbatim}
%  \LaTeX{} will see the \texttt{[v]} as an optional argument of
%  \env{gathered} and use it. In this case the test inside
%  \env{gathered} checks if it's a \texttt{t} or \texttt{b} and if
%  it's neither it'll choose \cs{vcenter} internally. So you get no
%  warning, only missing output. Another example, this time from the
%  \pkg{empheq} package used with its \opt{overload} option: If
%  preceding spaces are allowed, the input
%  \begin{verbatim}
%    \begin{gather}
%      [a] = [b]
%    \end{gather}
%  \end{verbatim}
%  results in the rather strange error message
%  \begin{verbatim}
%    ! Package keyval Error: a undefined.
%  \end{verbatim}
%
%  When using \cs{newcommand} etc.\ for defining commands and
%  environments with optional arguments, the peek ahead is done by
%  \cs{kernel@ifnextchar} (since \LaTeX{} release 2003/12/01, else
%  \cs{@ifnextchar}) and it is \emph{hardwired at definition time} by
%  \cs{@xargdef}. With the commands
%  \DescribeMacro{\MHPrecedingSpacesOff}\cs{MHPrecedingSpacesOff} and
%  \DescribeMacro{\MHPrecedingSpacesOn}\cs{MHPrecedingSpacesOn}
%  \pkg{mhsetup} provides an interface to define commands and
%  environments where the optional argument cannot have preceding
%  spaces. You simply wrap them around the definitions:
%  \begin{verbatim}
%  \MHPrecedingSpacesOff
%  \newenvironment*{test}[1][default]{Start, arg: (#1)}{Ending.}
%  \MHPrecedingSpacesOn
%  \begin{test}
%    [text]
%  \end{test}
%  \begin{test}[text]
%  \end{test}
%  \end{verbatim}
%  \MHPrecedingSpacesOff
%  \newenvironment*{test}[1][default]{Start, arg: (#1)}{Ending.}
%  \MHPrecedingSpacesOn
%  \begin{quote}
%  \begin{test}
%    [text]
%  \end{test}
%  \begin{test}[text]
%  \end{test}
%  \end{quote}
%  It is of somewhat limited use in commands (control words in \TeX{}
%  terminology), because \TeX{} discards the spaces. The exception is
%  \emph{control symbols} where \TeX{} obeys following spaces but
%  there are rather few of them available. All is not lost however.
%  In the \env{aligned} environment from \pkg{amsmath} (shown below)
%  a command is used as argument grabber.
%  \begin{verbatim}
%  \newenvironment{aligned}{%
%    \let\@testopt\alignsafe@testopt
%    \aligned@a
%  }{%
%    \crcr\egroup
%    \restorecolumn@
%    \egroup
%  }
%  \newcommand{\aligned@a}[1][c]{\start@aligned{#1}\m@ne}
%  \end{verbatim}
%  By applying our trick on the grabber function, we get a space
%  obeying version:
%  \begin{verbatim}
%  \MHPrecedingSpacesOff
%  \renewcommand*\aligned@a[1][c]{\start@aligned{#1}\m@ne}
%  \MHPrecedingSpacesOn
%  \end{verbatim}
%  This way a nested \env{aligned} environment is still safe from
%  empty first cells.
%
%  \StopEventually{}
%
%
%  \section{First bits of a new programming environment}
%
%
%    \begin{macrocode}
%<*package>
\ProvidesPackage{mhsetup}%
  [2007/12/03 v1.2 programming setup (MH)]
%    \end{macrocode}
%  \subsection{The new internal syntax}
%  \begin{macro}{\MHInternalSyntaxOn}
%  \begin{macro}{\MHInternalSyntaxOff}
%  Almost copy of \cs{InternalSyntaxOn}.
%    \begin{macrocode}
\def\MHInternalSyntaxOn{
  \edef\MHInternalSyntaxOff{%
  \catcode`\noexpand\~=\the\catcode`\~\relax
  \catcode`\noexpand\ =\the\catcode`\ \relax
  \catcode`\noexpand\^^I=\the\catcode`\^^I\relax
  \catcode`\noexpand\@=\the\catcode`\@\relax
  \catcode`\noexpand\:=\the\catcode`\:\relax
  \catcode`\noexpand\_=\the\catcode`\_\relax
  \endlinechar=\the\endlinechar\relax
  }%
  \catcode`\~=10\relax
  \catcode`\ =9\relax
  \catcode`\^^I=9\relax
  \makeatletter
  \catcode`\_=11\relax
  \catcode`\:=11\relax
  \endlinechar=` %
  \relax
}
\MHInternalSyntaxOn
\AtEndOfPackage{\MHInternalSyntaxOff}
%    \end{macrocode}
%  \end{macro}
%  \end{macro}
%  \subsection{Programming tools}
%
%  The whole idea is to provide programming tools that are convenient
%  but not yet widely available. I hope this'll be obsolete soon!
%
%  Firstly we setup a few helper functions.
%  \begin{macro}{\MH_let:NwN}
%  An alias for \cs{let}.
%    \begin{macrocode}
\let\MH_let:NwN \let
%    \end{macrocode}
%  \end{macro}
%  \begin{macro}{\MH_let:cN}
%  This one takes a \cs{csname}-\cs{endcsname} name and \cs{let}s it
%  to a single macro. We'll use this to setup our conditionals.
%    \begin{macrocode}
\def\MH_let:cN #1#2{
  \expandafter\MH_let:NwN \csname#1\endcsname#2}
%    \end{macrocode}
%  \end{macro}
%  \begin{macro}{\MH_let:cc}
%  This one has takes a \cs{csname}-\cs{endcsname} name and \cs{let}s
%  it to a another \cs{csname}-\cs{endcsname} name. To be used in
%  constructions with weird characters like \texttt{*} or alike in
%  them and can take a \cs{global} prefix if wanted (we want that
%  later on).
%    \begin{macrocode}
\def\MH_let:cc #1#2{
  \expandafter\MH_let:NwN\csname#1\expandafter\endcsname
  \csname#2\endcsname}
%    \end{macrocode}
%  \end{macro}
%
%  \begin{macro}{\MH_new_boolean:n}
%  \begin{macro}{\MH_set_boolean_F:n}
%  \begin{macro}{\MH_set_boolean_T:n}
%  \begin{macro}{\MH_if_boolean:nTF}
%  \begin{macro}{\MH_if_boolean:nT}
%  \begin{macro}{\MH_if_boolean:nF}
%  Sets up conditionals. For instance
%  \begin{quote}
%    \cs{MH_new_boolean:n} \marg{name}
%  \end{quote}
%  defines the boolean \meta{name} but also the conditional
%  \cs{if_boolean_\meta{name}:} to be used in the ordinary
%  \begin{tabbing}
%    \mbox{}\quad\= \kill
%    \cs{if_boolean_\meta{name}:}\\
%    \> \meta{true code} \\
%    \cs{else:} \\
%    \> \meta{false code} \\
%    \cs{fi:}
%  \end{tabbing}
%  There is also a more ``\LaTeX-like'' interface available by using
%  the commands
%  \begin{quote}
%    \cs{MH_if_boolean:nT}\marg{name}\marg{arg}
%  \end{quote}
%  which will execute the argument if the current value of the
%  boolean is `true' while
%  \begin{quote}
%  \cs{MH_if_boolean:nF}\marg{name}\marg{arg}
%  \end{quote}
%  is the equivalent with `false'. Finally we have
%  \begin{quote}
%  \cs{MH_if_boolean:nTF}\marg{name}\marg{true code}\marg{false code}.
%  \end{quote}
%  This is the interface I have used in this package.
%
%  Initially \cs{if_boolean_\meta{name}:} is `false'. This can be
%  changed by saying
%  \begin{tabbing}
%    \quad \= \LaTeX:\quad \= \kill
%    \> \TeX: \>   \cs{boolean_\meta{name}_true:} \quad \emph{or}\\
%    \> \LaTeX: \> \cs{MH_set_boolean_T:n}\marg{name}
%  \end{tabbing}
%  and changed back again by
%  \begin{tabbing}
%    \quad \= \LaTeX:\quad \= \kill
%    \> \TeX: \>   \cs{boolean_\meta{name}_false:} \quad \emph{or}\\
%    \> \LaTeX: \> \cs{MH_set_boolean_F:n}\marg{name}
%  \end{tabbing}
%
%  And yes, we're also using alternative names for \cs{else} and
%  \cs{fi} now. That way a simple search and replace will be all that
%  is needed for this package to be a certified \LaTeX3 package (well,
%  maybe a little more is needed, but not much).
%    \begin{macrocode}
\def\MH_new_boolean:n #1{
  \expandafter\@ifdefinable\csname if_boolean_#1:\endcsname{
    \@namedef{boolean_#1_true:}
      {\MH_let:cN{if_boolean_#1:}\iftrue}
    \@namedef{boolean_#1_false:}
      {\MH_let:cN{if_boolean_#1:}\iffalse}
    \@nameuse{boolean_#1_false:}%
  }
}
\def\MH_set_boolean_F:n #1{ \@nameuse{boolean_#1_false:} }
\def\MH_set_boolean_T:n #1{ \@nameuse{boolean_#1_true:} }
\def\MH_if_boolean:nTF #1{
  \@nameuse{if_boolean_#1:}
    \expandafter\@firstoftwo
  \else:
    \expandafter\@secondoftwo
  \fi:
}
\def\MH_if_boolean:nT #1{
  \@nameuse{if_boolean_#1:}
    \expandafter\@firstofone
  \else:
    \expandafter\@gobble
  \fi:
}
\def\MH_if_boolean:nF #1{
  \@nameuse{if_boolean_#1:}
    \expandafter\@gobble
  \else:
    \expandafter\@firstofone
  \fi:
}
%    \end{macrocode}
%  \end{macro}
%  \end{macro}
%  \end{macro}
%  \end{macro}
%  \end{macro}
%  \end{macro}
%
%  \begin{macro}{\if:w}
%  \changes{v1.0b}{2004/10/10}{Use different test}
%  \begin{macro}{\if_meaning:NN}
%  \changes{v1.0b}{2004/10/10}{Use different test}
%  \begin{macro}{\else:}
%  \changes{v1.0b}{2004/10/10}{Use different test}
%  \begin{macro}{\fi:}
%  \changes{v1.0b}{2004/10/10}{Use different test}
%  \begin{macro}{\if_num:w}
%  \changes{v1.0b}{2004/10/10}{Use different test}
%  \begin{macro}{\if_dim:w}
%  \changes{v1.0b}{2004/10/10}{Use different test}
%  \begin{macro}{\if_case:w}
%  \changes{v1.0b}{2004/10/10}{Use different test}
%  \begin{macro}{\or:}
%  \changes{v1.0b}{2004/10/10}{Use different test}
%  Copies of \TeX{} primitives.
%    \begin{macrocode}
\@ifundefined{if:w}{\MH_let:NwN \if:w =\if}{}
\@ifundefined{if_meaning:NN}{\MH_let:NwN \if_meaning:NN =\ifx}{}
\@ifundefined{else:}{\MH_let:NwN \else:=\else}{}
\@ifundefined{fi:}{\MH_let:NwN \fi:=\fi}{}
\@ifundefined{if_num:w}{\MH_let:NwN \if_num:w =\ifnum}{}
\@ifundefined{if_dim:w}{\MH_let:NwN \if_dim:w =\ifdim}{}
\@ifundefined{if_case:w}{\MH_let:NwN \if_case:w =\ifcase}{}
\@ifundefined{or:}{\MH_let:NwN \or:=\or}{}
%    \end{macrocode}
%  \end{macro}
%  \end{macro}
%  \end{macro}
%  \end{macro}
%  \end{macro}
%  \end{macro}
%  \end{macro}
%  \end{macro}
%  \begin{macro}{\MH_cs_to_str:N}
%  Strip off the backslash of a macro name.
%    \begin{macrocode}
\def\MH_cs_to_str:N {\expandafter\@gobble\string}
%    \end{macrocode}
%  \end{macro}
%
%  \begin{macro}{\MH_protected:}
%  \begin{macro}{\MH_setlength:dn}
%  \changes{v1.0a}{2004/08/23}{Fixed bug in \cs{MH_setlength:dn} so that
%  it now uses the \pkg{calc} definition of \cs{setlength}}
%  \begin{macro}{\MH_addtolength:dn}
%  \changes{v1.0a}{2004/08/23}{Fixed bug in \cs{MH_addtolength:dn} so that
%  it now uses the \pkg{calc} definition of \cs{addtolength}}
%  We might as well make use of some of the extended features from
%  \eTeX. We use \cs{dimexpr} for some simple calculations as it
%  saves a lot of the scanning that goes on inside \pkg{calc}. The
%  \cs{protected} primitive comes in handy when we want to declare a
%  robust command, that cannot be `robustified' with
%  \cs{DeclareRobustCommand}. If we don't have \eTeX{} we'll just let
%  our private macros be aliases for the less effective alternatives.
%    \begin{macrocode}
\@ifundefined{eTeXversion}
  {
    \MH_let:NwN \MH_protected:\relax
    \def\MH_setlength:dn{\setlength}
    \def\MH_addtolength:dn{\addtolength}
  }
  {
    \MH_let:NwN \MH_protected:\protected
    \def\MH_setlength:dn #1#2{#1=\dimexpr#2\relax\relax}
    \def\MH_addtolength:dn #1#2{\advance#1 \dimexpr#2\relax\relax}
  }
%    \end{macrocode}
%  \end{macro}
%  \end{macro}
%  \end{macro}

%  \begin{macro}{\MH_keyval_alias_with_addon:nnnn}
%  \begin{macro}{\MH_keyval_alias:nnn}
%  A way to make aliases with \pkg{keyval}. This will come in
%  handy later.
%    \begin{macrocode}
\def\MH_keyval_alias_with_addon:nnnn #1#2#3#4{
  \@namedef{KV@#1@#2}{\@nameuse{KV@#1@#3}#4}
  \@namedef{KV@#1@#2@default}{\@nameuse{KV@#1@#3@default}#4}}
\def\MH_keyval_alias:nnn #1#2#3{
  \MH_keyval_alias_with_addon:nnnn {#1}{#2}{#3}{}}
%    \end{macrocode}
%  \end{macro}
%  \end{macro}
%  \begin{macro}{\MH_use_choice_i:nnnn}
%  \begin{macro}{\MH_use_choice_ii:nnnn}
%  \begin{macro}{\MH_use_choice_iii:nnnn}
%  \begin{macro}{\MH_use_choice_iv:nnnn}
%  I need to be able to pick up individual arguments in a list of
%  four (similar to \cs{@secondoftwo}).
%    \begin{macrocode}
\def\MH_use_choice_i:nnnn #1#2#3#4{#1}
\def\MH_use_choice_ii:nnnn #1#2#3#4{#2}
\def\MH_use_choice_iii:nnnn #1#2#3#4{#3}
\def\MH_use_choice_iv:nnnn #1#2#3#4{#4}
%    \end{macrocode}
%  \end{macro}
%  \end{macro}
%  \end{macro}
%  \end{macro}
%  \begin{macro}{\MH_nospace_ifnextchar:Nnn}
%  \begin{macro}{\MH_nospace_nextchar:}
%  \begin{macro}{\MH_nospace_testopt:nn}
%  \begin{macro}{\MH_nospace_protected_testopt:n}
%  Scanning for the next character but disallow spaces.
%    \begin{macrocode}
\long\def\MH_nospace_ifnextchar:Nnn #1#2#3{
  \MH_let:NwN\reserved@d=~#1
  \def\reserved@a{#2}
  \def\reserved@b{#3}
  \futurelet\@let@token\MH_nospace_nextchar:
}
\def\MH_nospace_nextchar:{
  \if_meaning:NN \@let@token\reserved@d
    \MH_let:NwN \reserved@b\reserved@a
  \fi:
  \reserved@b
}
\long\def\MH_nospace_testopt:nn #1#2{
  \MH_nospace_ifnextchar:Nnn[
    {#1}
    {#1[{#2}]}
}
\def\MH_nospace_protected_testopt:n #1{
  \if_meaning:NN \protect\@typeset@protect
    \expandafter\MH_nospace_testopt:nn
  \else:
    \@x@protect#1
  \fi:
}
%    \end{macrocode}
%  \end{macro}
%  \end{macro}
%  \end{macro}
%  \end{macro}
%  \begin{macro}{\kernel@ifnextchar}
%  \begin{macro}{\MH_kernel_xargdef:nwwn}
%  \begin{macro}{\MH_nospace_xargdef:nwwn}
%  \begin{macro}{\MHPrecedingSpacesOff}
%  \begin{macro}{\MHPrecedingSpacesOn}
%  The code for the space sensitive peek ahead.
%    \begin{macrocode}
\@ifundefined{kernel@ifnextchar}
  {\MH_let:NwN \kernel@ifnextchar \@ifnextchar}
  {}
\MH_let:NwN \MH_kernel_xargdef:nwwn \@xargdef
\long\def\MH_nospace_xargdef:nwwn #1[#2][#3]#4{
  \@ifdefinable#1{
     \expandafter\def\expandafter#1\expandafter{
          \expandafter
          \MH_nospace_protected_testopt:n
          \expandafter
          #1
          \csname\string#1\endcsname
          {#3}}
       \expandafter\@yargdef
          \csname\string#1\endcsname
           \tw@
           {#2}
           {#4}}}
\providecommand*\MHPrecedingSpacesOff{
  \MH_let:NwN \@xargdef \MH_nospace_xargdef:nwwn
}
\providecommand*\MHPrecedingSpacesOn{
  \MH_let:NwN \@xargdef \MH_kernel_xargdef:nwwn
}
%    \end{macrocode}
%  \end{macro}
%  \end{macro}
%  \end{macro}
%  \end{macro}
%  \end{macro}
%
%  \begin{macro}{\MH_group_align_safe_begin:}
%  \begin{macro}{\MH_group_align_safe_end:}
%    \begin{macrocode}
\def \MH_group_align_safe_begin: {\iffalse{\fi\ifnum0=`}\fi}
\def \MH_group_align_safe_end:   {\ifnum0=`{}\fi}
%    \end{macrocode}
%  \end{macro}
%  \end{macro}
%
%    \begin{macrocode}
%</package>
%    \end{macrocode}
%
%  \Finale
\endinput

%        (quote the arguments according to the demands of your shell)
%
% Documentation:
%    (a) If mhsetup.drv is present:
%           latex mhsetup.drv
%    (b) Without mhsetup.drv:
%           latex mhsetup.dtx; ...
%    The class ltxdoc loads the configuration file ltxdoc.cfg
%    if available. Here you can specify further options, e.g.
%    use A4 as paper format:
%       \PassOptionsToClass{a4paper}{article}
%
%    Programm calls to get the documentation (example):
%       pdflatex mhsetup.dtx
%       makeindex -s gind.ist mhsetup.idx
%       pdflatex mhsetup.dtx
%       makeindex -s gind.ist mhsetup.idx
%       pdflatex mhsetup.dtx
%
% Installation:
%    TDS:tex/latex/mh/mhsetup.sty
%    TDS:doc/latex/mh/mhsetup.pdf
%    TDS:source/latex/mh/mhsetup.dtx
%
%<*ignore>
\begingroup
  \def\x{LaTeX2e}
\expandafter\endgroup
\ifcase 0\ifx\install y1\fi\expandafter
         \ifx\csname processbatchFile\endcsname\relax\else1\fi
         \ifx\fmtname\x\else 1\fi\relax
\else\csname fi\endcsname
%</ignore>
%<*install>
\input docstrip.tex
\Msg{************************************************************************}
\Msg{* Installation}
\Msg{* Package: mhsetup 2007/12/03 v1.2 mhsetup (MH)}
\Msg{************************************************************************}

\keepsilent
\askforoverwritefalse

\preamble

This is a generated file.

Copyright (C) 2002-2007 by Morten Hoegholm <mh.ctan@gmail.com>

This work may be distributed and/or modified under the
conditions of the LaTeX Project Public License, either
version 1.3 of this license or (at your option) any later
version. The latest version of this license is in
   http://www.latex-project.org/lppl.txt
and version 1.3 or later is part of all distributions of
LaTeX version 2005/12/01 or later.

This work has the LPPL maintenance status "maintained".

This Current Maintainer of this work is Morten Hoegholm.

This work consists of the main source file mhsetup.dtx
and the derived files
   mhsetup.sty, mhsetup.pdf, mhsetup.ins, mhsetup.drv.

\endpreamble

\generate{%
  \file{mhsetup.ins}{\from{mhsetup.dtx}{install}}%
  \file{mhsetup.drv}{\from{mhsetup.dtx}{driver}}%
  \usedir{tex/latex/mh}%
  \file{mhsetup.sty}{\from{mhsetup.dtx}{package}}%
}

\obeyspaces
\Msg{************************************************************************}
\Msg{*}
\Msg{* To finish the installation you have to move the following}
\Msg{* file into a directory searched by TeX:}
\Msg{*}
\Msg{*     mhsetup.sty}
\Msg{*}
\Msg{* To produce the documentation run the file `mhsetup.drv'}
\Msg{* through LaTeX.}
\Msg{*}
\Msg{* Happy TeXing!}
\Msg{*}
\Msg{************************************************************************}

\endbatchfile
%</install>
%<*ignore>
\fi
%</ignore>
%<*driver>
\NeedsTeXFormat{LaTeX2e}
\ProvidesFile{mhsetup.drv}%
  [2007/12/03 v1.2 programming setup (MH)]
\documentclass{ltxdoc}
\usepackage{mhsetup}

\addtolength\oddsidemargin{30pt}
\providecommand*\pkg[1]{\textsf{#1}}
\providecommand*\env[1]{\texttt{#1}}
\providecommand*\opt[1]{\texttt{#1}}
\def\MakePrivateLetters{%
  \makeatletter\catcode`\_=11\relax \catcode`\:=11\relax
}



\def\MHmarg#1{{\ttfamily\char`\{}\meta{#1}{\ttfamily\char`\}}\endgroup}
\DeclareRobustCommand\marg{\begingroup\MakePrivateLetters\MHmarg}%
\def\MHarg#1{{\ttfamily\char`\{#1\ttfamily\char`\}}\endgroup}
\DeclareRobustCommand\arg{\begingroup\MakePrivateLetters\MHarg}%
\def\MHcs#1{\texttt{\char`\\#1}\endgroup}
\DeclareRobustCommand\cs{\begingroup\MakePrivateLetters\MHcs}
\makeatletter
\providecommand*\eTeX{$\m@th\varepsilon$-\TeX}
\makeatother
\begin{document}
  \DocInput{mhsetup.dtx}
\end{document}
%</driver>
%  \fi
%
%  \CheckSum{297}
%
%  \GetFileInfo{mhsetup.drv}
%
%  \title{The \pkg{mhsetup} package\thanks{This package has version
%  number \fileversion, last revised on \filedate.}}
%  \author{Morten H\o gholm}
%  \date{\filedate}
%  \maketitle
%
%  \begin{abstract}
%    The \pkg{mhsetup} package provides tools for a \LaTeX{}
%    programming environment similar to the one described in
%    \texttt{expl3} on \textsc{ctan} although not as extensive. It is
%    a required part of both the \pkg{mathtools} and \pkg{empheq}
%    packages.
%  \end{abstract}
%
%  The description below was made before the extensive changes made to
%  the expl3 code available from the LaTeX Project website.
%
%  \section{The new internal syntax}
%
%  
%  The \LaTeX3 package \pkg{ldcsetup} defines the command
%  \cs{InternalSyntaxOn} which makes |_| and |:| letters and then
%  automatically restores the category codes at the end of the
%  package. This usually works fine but when you try to load
%  \pkg{amstext} you will experience that \TeX{} goes into an
%  infinite loop. Packages containing code like
%  |\@for\@tempa:=\@tempb\do{...}| will not work correctly either,
%  thus we provide an alternative version here with the pair of
%  commands
%  \DescribeMacro{\MHInternalSyntaxOn}\cs{MHInternalSyntaxOn} and
%  \DescribeMacro{\MHInternalSyntaxOff}\cs{MHInternalSyntaxOff}. They
%  are to be used only as a pair, because \cs{MHInternalSyntaxOn}
%  defines \cs{MHInternalSyntaxOff} so that it restores the category
%  codes correctly.
%
%
%  \section{Handling optional arguments}
%
%  The standard behavior of scanning for optional arguments in
%  \LaTeX{} allows any number of spaces preceding the optional
%  argument and that is not always good in math. For that reason
%  \pkg{amsmath} makes sure that commands like |\\| disallows spaces
%  before the optional argument but at the same time it fails to
%  provide ``safe'' environments. What would you expect from the
%  following input?
%  \begin{verbatim}
%  \[
%    \begin{gathered}
%      [v] = 100 \\
%      [t] = 200
%    \end{gathered}
%  \]
%  \end{verbatim}
%  \LaTeX{} will see the \texttt{[v]} as an optional argument of
%  \env{gathered} and use it. In this case the test inside
%  \env{gathered} checks if it's a \texttt{t} or \texttt{b} and if
%  it's neither it'll choose \cs{vcenter} internally. So you get no
%  warning, only missing output. Another example, this time from the
%  \pkg{empheq} package used with its \opt{overload} option: If
%  preceding spaces are allowed, the input
%  \begin{verbatim}
%    \begin{gather}
%      [a] = [b]
%    \end{gather}
%  \end{verbatim}
%  results in the rather strange error message
%  \begin{verbatim}
%    ! Package keyval Error: a undefined.
%  \end{verbatim}
%
%  When using \cs{newcommand} etc.\ for defining commands and
%  environments with optional arguments, the peek ahead is done by
%  \cs{kernel@ifnextchar} (since \LaTeX{} release 2003/12/01, else
%  \cs{@ifnextchar}) and it is \emph{hardwired at definition time} by
%  \cs{@xargdef}. With the commands
%  \DescribeMacro{\MHPrecedingSpacesOff}\cs{MHPrecedingSpacesOff} and
%  \DescribeMacro{\MHPrecedingSpacesOn}\cs{MHPrecedingSpacesOn}
%  \pkg{mhsetup} provides an interface to define commands and
%  environments where the optional argument cannot have preceding
%  spaces. You simply wrap them around the definitions:
%  \begin{verbatim}
%  \MHPrecedingSpacesOff
%  \newenvironment*{test}[1][default]{Start, arg: (#1)}{Ending.}
%  \MHPrecedingSpacesOn
%  \begin{test}
%    [text]
%  \end{test}
%  \begin{test}[text]
%  \end{test}
%  \end{verbatim}
%  \MHPrecedingSpacesOff
%  \newenvironment*{test}[1][default]{Start, arg: (#1)}{Ending.}
%  \MHPrecedingSpacesOn
%  \begin{quote}
%  \begin{test}
%    [text]
%  \end{test}
%  \begin{test}[text]
%  \end{test}
%  \end{quote}
%  It is of somewhat limited use in commands (control words in \TeX{}
%  terminology), because \TeX{} discards the spaces. The exception is
%  \emph{control symbols} where \TeX{} obeys following spaces but
%  there are rather few of them available. All is not lost however.
%  In the \env{aligned} environment from \pkg{amsmath} (shown below)
%  a command is used as argument grabber.
%  \begin{verbatim}
%  \newenvironment{aligned}{%
%    \let\@testopt\alignsafe@testopt
%    \aligned@a
%  }{%
%    \crcr\egroup
%    \restorecolumn@
%    \egroup
%  }
%  \newcommand{\aligned@a}[1][c]{\start@aligned{#1}\m@ne}
%  \end{verbatim}
%  By applying our trick on the grabber function, we get a space
%  obeying version:
%  \begin{verbatim}
%  \MHPrecedingSpacesOff
%  \renewcommand*\aligned@a[1][c]{\start@aligned{#1}\m@ne}
%  \MHPrecedingSpacesOn
%  \end{verbatim}
%  This way a nested \env{aligned} environment is still safe from
%  empty first cells.
%
%  \StopEventually{}
%
%
%  \section{First bits of a new programming environment}
%
%
%    \begin{macrocode}
%<*package>
\ProvidesPackage{mhsetup}%
  [2007/12/03 v1.2 programming setup (MH)]
%    \end{macrocode}
%  \subsection{The new internal syntax}
%  \begin{macro}{\MHInternalSyntaxOn}
%  \begin{macro}{\MHInternalSyntaxOff}
%  Almost copy of \cs{InternalSyntaxOn}.
%    \begin{macrocode}
\def\MHInternalSyntaxOn{
  \edef\MHInternalSyntaxOff{%
  \catcode`\noexpand\~=\the\catcode`\~\relax
  \catcode`\noexpand\ =\the\catcode`\ \relax
  \catcode`\noexpand\^^I=\the\catcode`\^^I\relax
  \catcode`\noexpand\@=\the\catcode`\@\relax
  \catcode`\noexpand\:=\the\catcode`\:\relax
  \catcode`\noexpand\_=\the\catcode`\_\relax
  \endlinechar=\the\endlinechar\relax
  }%
  \catcode`\~=10\relax
  \catcode`\ =9\relax
  \catcode`\^^I=9\relax
  \makeatletter
  \catcode`\_=11\relax
  \catcode`\:=11\relax
  \endlinechar=` %
  \relax
}
\MHInternalSyntaxOn
\AtEndOfPackage{\MHInternalSyntaxOff}
%    \end{macrocode}
%  \end{macro}
%  \end{macro}
%  \subsection{Programming tools}
%
%  The whole idea is to provide programming tools that are convenient
%  but not yet widely available. I hope this'll be obsolete soon!
%
%  Firstly we setup a few helper functions.
%  \begin{macro}{\MH_let:NwN}
%  An alias for \cs{let}.
%    \begin{macrocode}
\let\MH_let:NwN \let
%    \end{macrocode}
%  \end{macro}
%  \begin{macro}{\MH_let:cN}
%  This one takes a \cs{csname}-\cs{endcsname} name and \cs{let}s it
%  to a single macro. We'll use this to setup our conditionals.
%    \begin{macrocode}
\def\MH_let:cN #1#2{
  \expandafter\MH_let:NwN \csname#1\endcsname#2}
%    \end{macrocode}
%  \end{macro}
%  \begin{macro}{\MH_let:cc}
%  This one has takes a \cs{csname}-\cs{endcsname} name and \cs{let}s
%  it to a another \cs{csname}-\cs{endcsname} name. To be used in
%  constructions with weird characters like \texttt{*} or alike in
%  them and can take a \cs{global} prefix if wanted (we want that
%  later on).
%    \begin{macrocode}
\def\MH_let:cc #1#2{
  \expandafter\MH_let:NwN\csname#1\expandafter\endcsname
  \csname#2\endcsname}
%    \end{macrocode}
%  \end{macro}
%
%  \begin{macro}{\MH_new_boolean:n}
%  \begin{macro}{\MH_set_boolean_F:n}
%  \begin{macro}{\MH_set_boolean_T:n}
%  \begin{macro}{\MH_if_boolean:nTF}
%  \begin{macro}{\MH_if_boolean:nT}
%  \begin{macro}{\MH_if_boolean:nF}
%  Sets up conditionals. For instance
%  \begin{quote}
%    \cs{MH_new_boolean:n} \marg{name}
%  \end{quote}
%  defines the boolean \meta{name} but also the conditional
%  \cs{if_boolean_\meta{name}:} to be used in the ordinary
%  \begin{tabbing}
%    \mbox{}\quad\= \kill
%    \cs{if_boolean_\meta{name}:}\\
%    \> \meta{true code} \\
%    \cs{else:} \\
%    \> \meta{false code} \\
%    \cs{fi:}
%  \end{tabbing}
%  There is also a more ``\LaTeX-like'' interface available by using
%  the commands
%  \begin{quote}
%    \cs{MH_if_boolean:nT}\marg{name}\marg{arg}
%  \end{quote}
%  which will execute the argument if the current value of the
%  boolean is `true' while
%  \begin{quote}
%  \cs{MH_if_boolean:nF}\marg{name}\marg{arg}
%  \end{quote}
%  is the equivalent with `false'. Finally we have
%  \begin{quote}
%  \cs{MH_if_boolean:nTF}\marg{name}\marg{true code}\marg{false code}.
%  \end{quote}
%  This is the interface I have used in this package.
%
%  Initially \cs{if_boolean_\meta{name}:} is `false'. This can be
%  changed by saying
%  \begin{tabbing}
%    \quad \= \LaTeX:\quad \= \kill
%    \> \TeX: \>   \cs{boolean_\meta{name}_true:} \quad \emph{or}\\
%    \> \LaTeX: \> \cs{MH_set_boolean_T:n}\marg{name}
%  \end{tabbing}
%  and changed back again by
%  \begin{tabbing}
%    \quad \= \LaTeX:\quad \= \kill
%    \> \TeX: \>   \cs{boolean_\meta{name}_false:} \quad \emph{or}\\
%    \> \LaTeX: \> \cs{MH_set_boolean_F:n}\marg{name}
%  \end{tabbing}
%
%  And yes, we're also using alternative names for \cs{else} and
%  \cs{fi} now. That way a simple search and replace will be all that
%  is needed for this package to be a certified \LaTeX3 package (well,
%  maybe a little more is needed, but not much).
%    \begin{macrocode}
\def\MH_new_boolean:n #1{
  \expandafter\@ifdefinable\csname if_boolean_#1:\endcsname{
    \@namedef{boolean_#1_true:}
      {\MH_let:cN{if_boolean_#1:}\iftrue}
    \@namedef{boolean_#1_false:}
      {\MH_let:cN{if_boolean_#1:}\iffalse}
    \@nameuse{boolean_#1_false:}%
  }
}
\def\MH_set_boolean_F:n #1{ \@nameuse{boolean_#1_false:} }
\def\MH_set_boolean_T:n #1{ \@nameuse{boolean_#1_true:} }
\def\MH_if_boolean:nTF #1{
  \@nameuse{if_boolean_#1:}
    \expandafter\@firstoftwo
  \else:
    \expandafter\@secondoftwo
  \fi:
}
\def\MH_if_boolean:nT #1{
  \@nameuse{if_boolean_#1:}
    \expandafter\@firstofone
  \else:
    \expandafter\@gobble
  \fi:
}
\def\MH_if_boolean:nF #1{
  \@nameuse{if_boolean_#1:}
    \expandafter\@gobble
  \else:
    \expandafter\@firstofone
  \fi:
}
%    \end{macrocode}
%  \end{macro}
%  \end{macro}
%  \end{macro}
%  \end{macro}
%  \end{macro}
%  \end{macro}
%
%  \begin{macro}{\if:w}
%  \changes{v1.0b}{2004/10/10}{Use different test}
%  \begin{macro}{\if_meaning:NN}
%  \changes{v1.0b}{2004/10/10}{Use different test}
%  \begin{macro}{\else:}
%  \changes{v1.0b}{2004/10/10}{Use different test}
%  \begin{macro}{\fi:}
%  \changes{v1.0b}{2004/10/10}{Use different test}
%  \begin{macro}{\if_num:w}
%  \changes{v1.0b}{2004/10/10}{Use different test}
%  \begin{macro}{\if_dim:w}
%  \changes{v1.0b}{2004/10/10}{Use different test}
%  \begin{macro}{\if_case:w}
%  \changes{v1.0b}{2004/10/10}{Use different test}
%  \begin{macro}{\or:}
%  \changes{v1.0b}{2004/10/10}{Use different test}
%  Copies of \TeX{} primitives.
%    \begin{macrocode}
\@ifundefined{if:w}{\MH_let:NwN \if:w =\if}{}
\@ifundefined{if_meaning:NN}{\MH_let:NwN \if_meaning:NN =\ifx}{}
\@ifundefined{else:}{\MH_let:NwN \else:=\else}{}
\@ifundefined{fi:}{\MH_let:NwN \fi:=\fi}{}
\@ifundefined{if_num:w}{\MH_let:NwN \if_num:w =\ifnum}{}
\@ifundefined{if_dim:w}{\MH_let:NwN \if_dim:w =\ifdim}{}
\@ifundefined{if_case:w}{\MH_let:NwN \if_case:w =\ifcase}{}
\@ifundefined{or:}{\MH_let:NwN \or:=\or}{}
%    \end{macrocode}
%  \end{macro}
%  \end{macro}
%  \end{macro}
%  \end{macro}
%  \end{macro}
%  \end{macro}
%  \end{macro}
%  \end{macro}
%  \begin{macro}{\MH_cs_to_str:N}
%  Strip off the backslash of a macro name.
%    \begin{macrocode}
\def\MH_cs_to_str:N {\expandafter\@gobble\string}
%    \end{macrocode}
%  \end{macro}
%
%  \begin{macro}{\MH_protected:}
%  \begin{macro}{\MH_setlength:dn}
%  \changes{v1.0a}{2004/08/23}{Fixed bug in \cs{MH_setlength:dn} so that
%  it now uses the \pkg{calc} definition of \cs{setlength}}
%  \begin{macro}{\MH_addtolength:dn}
%  \changes{v1.0a}{2004/08/23}{Fixed bug in \cs{MH_addtolength:dn} so that
%  it now uses the \pkg{calc} definition of \cs{addtolength}}
%  We might as well make use of some of the extended features from
%  \eTeX. We use \cs{dimexpr} for some simple calculations as it
%  saves a lot of the scanning that goes on inside \pkg{calc}. The
%  \cs{protected} primitive comes in handy when we want to declare a
%  robust command, that cannot be `robustified' with
%  \cs{DeclareRobustCommand}. If we don't have \eTeX{} we'll just let
%  our private macros be aliases for the less effective alternatives.
%    \begin{macrocode}
\@ifundefined{eTeXversion}
  {
    \MH_let:NwN \MH_protected:\relax
    \def\MH_setlength:dn{\setlength}
    \def\MH_addtolength:dn{\addtolength}
  }
  {
    \MH_let:NwN \MH_protected:\protected
    \def\MH_setlength:dn #1#2{#1=\dimexpr#2\relax\relax}
    \def\MH_addtolength:dn #1#2{\advance#1 \dimexpr#2\relax\relax}
  }
%    \end{macrocode}
%  \end{macro}
%  \end{macro}
%  \end{macro}

%  \begin{macro}{\MH_keyval_alias_with_addon:nnnn}
%  \begin{macro}{\MH_keyval_alias:nnn}
%  A way to make aliases with \pkg{keyval}. This will come in
%  handy later.
%    \begin{macrocode}
\def\MH_keyval_alias_with_addon:nnnn #1#2#3#4{
  \@namedef{KV@#1@#2}{\@nameuse{KV@#1@#3}#4}
  \@namedef{KV@#1@#2@default}{\@nameuse{KV@#1@#3@default}#4}}
\def\MH_keyval_alias:nnn #1#2#3{
  \MH_keyval_alias_with_addon:nnnn {#1}{#2}{#3}{}}
%    \end{macrocode}
%  \end{macro}
%  \end{macro}
%  \begin{macro}{\MH_use_choice_i:nnnn}
%  \begin{macro}{\MH_use_choice_ii:nnnn}
%  \begin{macro}{\MH_use_choice_iii:nnnn}
%  \begin{macro}{\MH_use_choice_iv:nnnn}
%  I need to be able to pick up individual arguments in a list of
%  four (similar to \cs{@secondoftwo}).
%    \begin{macrocode}
\def\MH_use_choice_i:nnnn #1#2#3#4{#1}
\def\MH_use_choice_ii:nnnn #1#2#3#4{#2}
\def\MH_use_choice_iii:nnnn #1#2#3#4{#3}
\def\MH_use_choice_iv:nnnn #1#2#3#4{#4}
%    \end{macrocode}
%  \end{macro}
%  \end{macro}
%  \end{macro}
%  \end{macro}
%  \begin{macro}{\MH_nospace_ifnextchar:Nnn}
%  \begin{macro}{\MH_nospace_nextchar:}
%  \begin{macro}{\MH_nospace_testopt:nn}
%  \begin{macro}{\MH_nospace_protected_testopt:n}
%  Scanning for the next character but disallow spaces.
%    \begin{macrocode}
\long\def\MH_nospace_ifnextchar:Nnn #1#2#3{
  \MH_let:NwN\reserved@d=~#1
  \def\reserved@a{#2}
  \def\reserved@b{#3}
  \futurelet\@let@token\MH_nospace_nextchar:
}
\def\MH_nospace_nextchar:{
  \if_meaning:NN \@let@token\reserved@d
    \MH_let:NwN \reserved@b\reserved@a
  \fi:
  \reserved@b
}
\long\def\MH_nospace_testopt:nn #1#2{
  \MH_nospace_ifnextchar:Nnn[
    {#1}
    {#1[{#2}]}
}
\def\MH_nospace_protected_testopt:n #1{
  \if_meaning:NN \protect\@typeset@protect
    \expandafter\MH_nospace_testopt:nn
  \else:
    \@x@protect#1
  \fi:
}
%    \end{macrocode}
%  \end{macro}
%  \end{macro}
%  \end{macro}
%  \end{macro}
%  \begin{macro}{\kernel@ifnextchar}
%  \begin{macro}{\MH_kernel_xargdef:nwwn}
%  \begin{macro}{\MH_nospace_xargdef:nwwn}
%  \begin{macro}{\MHPrecedingSpacesOff}
%  \begin{macro}{\MHPrecedingSpacesOn}
%  The code for the space sensitive peek ahead.
%    \begin{macrocode}
\@ifundefined{kernel@ifnextchar}
  {\MH_let:NwN \kernel@ifnextchar \@ifnextchar}
  {}
\MH_let:NwN \MH_kernel_xargdef:nwwn \@xargdef
\long\def\MH_nospace_xargdef:nwwn #1[#2][#3]#4{
  \@ifdefinable#1{
     \expandafter\def\expandafter#1\expandafter{
          \expandafter
          \MH_nospace_protected_testopt:n
          \expandafter
          #1
          \csname\string#1\endcsname
          {#3}}
       \expandafter\@yargdef
          \csname\string#1\endcsname
           \tw@
           {#2}
           {#4}}}
\providecommand*\MHPrecedingSpacesOff{
  \MH_let:NwN \@xargdef \MH_nospace_xargdef:nwwn
}
\providecommand*\MHPrecedingSpacesOn{
  \MH_let:NwN \@xargdef \MH_kernel_xargdef:nwwn
}
%    \end{macrocode}
%  \end{macro}
%  \end{macro}
%  \end{macro}
%  \end{macro}
%  \end{macro}
%
%  \begin{macro}{\MH_group_align_safe_begin:}
%  \begin{macro}{\MH_group_align_safe_end:}
%    \begin{macrocode}
\def \MH_group_align_safe_begin: {\iffalse{\fi\ifnum0=`}\fi}
\def \MH_group_align_safe_end:   {\ifnum0=`{}\fi}
%    \end{macrocode}
%  \end{macro}
%  \end{macro}
%
%    \begin{macrocode}
%</package>
%    \end{macrocode}
%
%  \Finale
\endinput

%        (quote the arguments according to the demands of your shell)
%
% Documentation:
%    (a) If mhsetup.drv is present:
%           latex mhsetup.drv
%    (b) Without mhsetup.drv:
%           latex mhsetup.dtx; ...
%    The class ltxdoc loads the configuration file ltxdoc.cfg
%    if available. Here you can specify further options, e.g.
%    use A4 as paper format:
%       \PassOptionsToClass{a4paper}{article}
%
%    Programm calls to get the documentation (example):
%       pdflatex mhsetup.dtx
%       makeindex -s gind.ist mhsetup.idx
%       pdflatex mhsetup.dtx
%       makeindex -s gind.ist mhsetup.idx
%       pdflatex mhsetup.dtx
%
% Installation:
%    TDS:tex/latex/mh/mhsetup.sty
%    TDS:doc/latex/mh/mhsetup.pdf
%    TDS:source/latex/mh/mhsetup.dtx
%
%<*ignore>
\begingroup
  \def\x{LaTeX2e}
\expandafter\endgroup
\ifcase 0\ifx\install y1\fi\expandafter
         \ifx\csname processbatchFile\endcsname\relax\else1\fi
         \ifx\fmtname\x\else 1\fi\relax
\else\csname fi\endcsname
%</ignore>
%<*install>
\input docstrip.tex
\Msg{************************************************************************}
\Msg{* Installation}
\Msg{* Package: mhsetup 2007/12/03 v1.2 mhsetup (MH)}
\Msg{************************************************************************}

\keepsilent
\askforoverwritefalse

\preamble

This is a generated file.

Copyright (C) 2002-2007 by Morten Hoegholm <mh.ctan@gmail.com>

This work may be distributed and/or modified under the
conditions of the LaTeX Project Public License, either
version 1.3 of this license or (at your option) any later
version. The latest version of this license is in
   http://www.latex-project.org/lppl.txt
and version 1.3 or later is part of all distributions of
LaTeX version 2005/12/01 or later.

This work has the LPPL maintenance status "maintained".

This Current Maintainer of this work is Morten Hoegholm.

This work consists of the main source file mhsetup.dtx
and the derived files
   mhsetup.sty, mhsetup.pdf, mhsetup.ins, mhsetup.drv.

\endpreamble

\generate{%
  \file{mhsetup.ins}{\from{mhsetup.dtx}{install}}%
  \file{mhsetup.drv}{\from{mhsetup.dtx}{driver}}%
  \usedir{tex/latex/mh}%
  \file{mhsetup.sty}{\from{mhsetup.dtx}{package}}%
}

\obeyspaces
\Msg{************************************************************************}
\Msg{*}
\Msg{* To finish the installation you have to move the following}
\Msg{* file into a directory searched by TeX:}
\Msg{*}
\Msg{*     mhsetup.sty}
\Msg{*}
\Msg{* To produce the documentation run the file `mhsetup.drv'}
\Msg{* through LaTeX.}
\Msg{*}
\Msg{* Happy TeXing!}
\Msg{*}
\Msg{************************************************************************}

\endbatchfile
%</install>
%<*ignore>
\fi
%</ignore>
%<*driver>
\NeedsTeXFormat{LaTeX2e}
\ProvidesFile{mhsetup.drv}%
  [2007/12/03 v1.2 programming setup (MH)]
\documentclass{ltxdoc}
\usepackage{mhsetup}

\addtolength\oddsidemargin{30pt}
\providecommand*\pkg[1]{\textsf{#1}}
\providecommand*\env[1]{\texttt{#1}}
\providecommand*\opt[1]{\texttt{#1}}
\def\MakePrivateLetters{%
  \makeatletter\catcode`\_=11\relax \catcode`\:=11\relax
}



\def\MHmarg#1{{\ttfamily\char`\{}\meta{#1}{\ttfamily\char`\}}\endgroup}
\DeclareRobustCommand\marg{\begingroup\MakePrivateLetters\MHmarg}%
\def\MHarg#1{{\ttfamily\char`\{#1\ttfamily\char`\}}\endgroup}
\DeclareRobustCommand\arg{\begingroup\MakePrivateLetters\MHarg}%
\def\MHcs#1{\texttt{\char`\\#1}\endgroup}
\DeclareRobustCommand\cs{\begingroup\MakePrivateLetters\MHcs}
\makeatletter
\providecommand*\eTeX{$\m@th\varepsilon$-\TeX}
\makeatother
\begin{document}
  \DocInput{mhsetup.dtx}
\end{document}
%</driver>
%  \fi
%
%  \CheckSum{297}
%
%  \GetFileInfo{mhsetup.drv}
%
%  \title{The \pkg{mhsetup} package\thanks{This package has version
%  number \fileversion, last revised on \filedate.}}
%  \author{Morten H\o gholm}
%  \date{\filedate}
%  \maketitle
%
%  \begin{abstract}
%    The \pkg{mhsetup} package provides tools for a \LaTeX{}
%    programming environment similar to the one described in
%    \texttt{expl3} on \textsc{ctan} although not as extensive. It is
%    a required part of both the \pkg{mathtools} and \pkg{empheq}
%    packages.
%  \end{abstract}
%
%  The description below was made before the extensive changes made to
%  the expl3 code available from the LaTeX Project website.
%
%  \section{The new internal syntax}
%
%  
%  The \LaTeX3 package \pkg{ldcsetup} defines the command
%  \cs{InternalSyntaxOn} which makes |_| and |:| letters and then
%  automatically restores the category codes at the end of the
%  package. This usually works fine but when you try to load
%  \pkg{amstext} you will experience that \TeX{} goes into an
%  infinite loop. Packages containing code like
%  |\@for\@tempa:=\@tempb\do{...}| will not work correctly either,
%  thus we provide an alternative version here with the pair of
%  commands
%  \DescribeMacro{\MHInternalSyntaxOn}\cs{MHInternalSyntaxOn} and
%  \DescribeMacro{\MHInternalSyntaxOff}\cs{MHInternalSyntaxOff}. They
%  are to be used only as a pair, because \cs{MHInternalSyntaxOn}
%  defines \cs{MHInternalSyntaxOff} so that it restores the category
%  codes correctly.
%
%
%  \section{Handling optional arguments}
%
%  The standard behavior of scanning for optional arguments in
%  \LaTeX{} allows any number of spaces preceding the optional
%  argument and that is not always good in math. For that reason
%  \pkg{amsmath} makes sure that commands like |\\| disallows spaces
%  before the optional argument but at the same time it fails to
%  provide ``safe'' environments. What would you expect from the
%  following input?
%  \begin{verbatim}
%  \[
%    \begin{gathered}
%      [v] = 100 \\
%      [t] = 200
%    \end{gathered}
%  \]
%  \end{verbatim}
%  \LaTeX{} will see the \texttt{[v]} as an optional argument of
%  \env{gathered} and use it. In this case the test inside
%  \env{gathered} checks if it's a \texttt{t} or \texttt{b} and if
%  it's neither it'll choose \cs{vcenter} internally. So you get no
%  warning, only missing output. Another example, this time from the
%  \pkg{empheq} package used with its \opt{overload} option: If
%  preceding spaces are allowed, the input
%  \begin{verbatim}
%    \begin{gather}
%      [a] = [b]
%    \end{gather}
%  \end{verbatim}
%  results in the rather strange error message
%  \begin{verbatim}
%    ! Package keyval Error: a undefined.
%  \end{verbatim}
%
%  When using \cs{newcommand} etc.\ for defining commands and
%  environments with optional arguments, the peek ahead is done by
%  \cs{kernel@ifnextchar} (since \LaTeX{} release 2003/12/01, else
%  \cs{@ifnextchar}) and it is \emph{hardwired at definition time} by
%  \cs{@xargdef}. With the commands
%  \DescribeMacro{\MHPrecedingSpacesOff}\cs{MHPrecedingSpacesOff} and
%  \DescribeMacro{\MHPrecedingSpacesOn}\cs{MHPrecedingSpacesOn}
%  \pkg{mhsetup} provides an interface to define commands and
%  environments where the optional argument cannot have preceding
%  spaces. You simply wrap them around the definitions:
%  \begin{verbatim}
%  \MHPrecedingSpacesOff
%  \newenvironment*{test}[1][default]{Start, arg: (#1)}{Ending.}
%  \MHPrecedingSpacesOn
%  \begin{test}
%    [text]
%  \end{test}
%  \begin{test}[text]
%  \end{test}
%  \end{verbatim}
%  \MHPrecedingSpacesOff
%  \newenvironment*{test}[1][default]{Start, arg: (#1)}{Ending.}
%  \MHPrecedingSpacesOn
%  \begin{quote}
%  \begin{test}
%    [text]
%  \end{test}
%  \begin{test}[text]
%  \end{test}
%  \end{quote}
%  It is of somewhat limited use in commands (control words in \TeX{}
%  terminology), because \TeX{} discards the spaces. The exception is
%  \emph{control symbols} where \TeX{} obeys following spaces but
%  there are rather few of them available. All is not lost however.
%  In the \env{aligned} environment from \pkg{amsmath} (shown below)
%  a command is used as argument grabber.
%  \begin{verbatim}
%  \newenvironment{aligned}{%
%    \let\@testopt\alignsafe@testopt
%    \aligned@a
%  }{%
%    \crcr\egroup
%    \restorecolumn@
%    \egroup
%  }
%  \newcommand{\aligned@a}[1][c]{\start@aligned{#1}\m@ne}
%  \end{verbatim}
%  By applying our trick on the grabber function, we get a space
%  obeying version:
%  \begin{verbatim}
%  \MHPrecedingSpacesOff
%  \renewcommand*\aligned@a[1][c]{\start@aligned{#1}\m@ne}
%  \MHPrecedingSpacesOn
%  \end{verbatim}
%  This way a nested \env{aligned} environment is still safe from
%  empty first cells.
%
%  \StopEventually{}
%
%
%  \section{First bits of a new programming environment}
%
%
%    \begin{macrocode}
%<*package>
\ProvidesPackage{mhsetup}%
  [2007/12/03 v1.2 programming setup (MH)]
%    \end{macrocode}
%  \subsection{The new internal syntax}
%  \begin{macro}{\MHInternalSyntaxOn}
%  \begin{macro}{\MHInternalSyntaxOff}
%  Almost copy of \cs{InternalSyntaxOn}.
%    \begin{macrocode}
\def\MHInternalSyntaxOn{
  \edef\MHInternalSyntaxOff{%
  \catcode`\noexpand\~=\the\catcode`\~\relax
  \catcode`\noexpand\ =\the\catcode`\ \relax
  \catcode`\noexpand\^^I=\the\catcode`\^^I\relax
  \catcode`\noexpand\@=\the\catcode`\@\relax
  \catcode`\noexpand\:=\the\catcode`\:\relax
  \catcode`\noexpand\_=\the\catcode`\_\relax
  \endlinechar=\the\endlinechar\relax
  }%
  \catcode`\~=10\relax
  \catcode`\ =9\relax
  \catcode`\^^I=9\relax
  \makeatletter
  \catcode`\_=11\relax
  \catcode`\:=11\relax
  \endlinechar=` %
  \relax
}
\MHInternalSyntaxOn
\AtEndOfPackage{\MHInternalSyntaxOff}
%    \end{macrocode}
%  \end{macro}
%  \end{macro}
%  \subsection{Programming tools}
%
%  The whole idea is to provide programming tools that are convenient
%  but not yet widely available. I hope this'll be obsolete soon!
%
%  Firstly we setup a few helper functions.
%  \begin{macro}{\MH_let:NwN}
%  An alias for \cs{let}.
%    \begin{macrocode}
\let\MH_let:NwN \let
%    \end{macrocode}
%  \end{macro}
%  \begin{macro}{\MH_let:cN}
%  This one takes a \cs{csname}-\cs{endcsname} name and \cs{let}s it
%  to a single macro. We'll use this to setup our conditionals.
%    \begin{macrocode}
\def\MH_let:cN #1#2{
  \expandafter\MH_let:NwN \csname#1\endcsname#2}
%    \end{macrocode}
%  \end{macro}
%  \begin{macro}{\MH_let:cc}
%  This one has takes a \cs{csname}-\cs{endcsname} name and \cs{let}s
%  it to a another \cs{csname}-\cs{endcsname} name. To be used in
%  constructions with weird characters like \texttt{*} or alike in
%  them and can take a \cs{global} prefix if wanted (we want that
%  later on).
%    \begin{macrocode}
\def\MH_let:cc #1#2{
  \expandafter\MH_let:NwN\csname#1\expandafter\endcsname
  \csname#2\endcsname}
%    \end{macrocode}
%  \end{macro}
%
%  \begin{macro}{\MH_new_boolean:n}
%  \begin{macro}{\MH_set_boolean_F:n}
%  \begin{macro}{\MH_set_boolean_T:n}
%  \begin{macro}{\MH_if_boolean:nTF}
%  \begin{macro}{\MH_if_boolean:nT}
%  \begin{macro}{\MH_if_boolean:nF}
%  Sets up conditionals. For instance
%  \begin{quote}
%    \cs{MH_new_boolean:n} \marg{name}
%  \end{quote}
%  defines the boolean \meta{name} but also the conditional
%  \cs{if_boolean_\meta{name}:} to be used in the ordinary
%  \begin{tabbing}
%    \mbox{}\quad\= \kill
%    \cs{if_boolean_\meta{name}:}\\
%    \> \meta{true code} \\
%    \cs{else:} \\
%    \> \meta{false code} \\
%    \cs{fi:}
%  \end{tabbing}
%  There is also a more ``\LaTeX-like'' interface available by using
%  the commands
%  \begin{quote}
%    \cs{MH_if_boolean:nT}\marg{name}\marg{arg}
%  \end{quote}
%  which will execute the argument if the current value of the
%  boolean is `true' while
%  \begin{quote}
%  \cs{MH_if_boolean:nF}\marg{name}\marg{arg}
%  \end{quote}
%  is the equivalent with `false'. Finally we have
%  \begin{quote}
%  \cs{MH_if_boolean:nTF}\marg{name}\marg{true code}\marg{false code}.
%  \end{quote}
%  This is the interface I have used in this package.
%
%  Initially \cs{if_boolean_\meta{name}:} is `false'. This can be
%  changed by saying
%  \begin{tabbing}
%    \quad \= \LaTeX:\quad \= \kill
%    \> \TeX: \>   \cs{boolean_\meta{name}_true:} \quad \emph{or}\\
%    \> \LaTeX: \> \cs{MH_set_boolean_T:n}\marg{name}
%  \end{tabbing}
%  and changed back again by
%  \begin{tabbing}
%    \quad \= \LaTeX:\quad \= \kill
%    \> \TeX: \>   \cs{boolean_\meta{name}_false:} \quad \emph{or}\\
%    \> \LaTeX: \> \cs{MH_set_boolean_F:n}\marg{name}
%  \end{tabbing}
%
%  And yes, we're also using alternative names for \cs{else} and
%  \cs{fi} now. That way a simple search and replace will be all that
%  is needed for this package to be a certified \LaTeX3 package (well,
%  maybe a little more is needed, but not much).
%    \begin{macrocode}
\def\MH_new_boolean:n #1{
  \expandafter\@ifdefinable\csname if_boolean_#1:\endcsname{
    \@namedef{boolean_#1_true:}
      {\MH_let:cN{if_boolean_#1:}\iftrue}
    \@namedef{boolean_#1_false:}
      {\MH_let:cN{if_boolean_#1:}\iffalse}
    \@nameuse{boolean_#1_false:}%
  }
}
\def\MH_set_boolean_F:n #1{ \@nameuse{boolean_#1_false:} }
\def\MH_set_boolean_T:n #1{ \@nameuse{boolean_#1_true:} }
\def\MH_if_boolean:nTF #1{
  \@nameuse{if_boolean_#1:}
    \expandafter\@firstoftwo
  \else:
    \expandafter\@secondoftwo
  \fi:
}
\def\MH_if_boolean:nT #1{
  \@nameuse{if_boolean_#1:}
    \expandafter\@firstofone
  \else:
    \expandafter\@gobble
  \fi:
}
\def\MH_if_boolean:nF #1{
  \@nameuse{if_boolean_#1:}
    \expandafter\@gobble
  \else:
    \expandafter\@firstofone
  \fi:
}
%    \end{macrocode}
%  \end{macro}
%  \end{macro}
%  \end{macro}
%  \end{macro}
%  \end{macro}
%  \end{macro}
%
%  \begin{macro}{\if:w}
%  \changes{v1.0b}{2004/10/10}{Use different test}
%  \begin{macro}{\if_meaning:NN}
%  \changes{v1.0b}{2004/10/10}{Use different test}
%  \begin{macro}{\else:}
%  \changes{v1.0b}{2004/10/10}{Use different test}
%  \begin{macro}{\fi:}
%  \changes{v1.0b}{2004/10/10}{Use different test}
%  \begin{macro}{\if_num:w}
%  \changes{v1.0b}{2004/10/10}{Use different test}
%  \begin{macro}{\if_dim:w}
%  \changes{v1.0b}{2004/10/10}{Use different test}
%  \begin{macro}{\if_case:w}
%  \changes{v1.0b}{2004/10/10}{Use different test}
%  \begin{macro}{\or:}
%  \changes{v1.0b}{2004/10/10}{Use different test}
%  Copies of \TeX{} primitives.
%    \begin{macrocode}
\@ifundefined{if:w}{\MH_let:NwN \if:w =\if}{}
\@ifundefined{if_meaning:NN}{\MH_let:NwN \if_meaning:NN =\ifx}{}
\@ifundefined{else:}{\MH_let:NwN \else:=\else}{}
\@ifundefined{fi:}{\MH_let:NwN \fi:=\fi}{}
\@ifundefined{if_num:w}{\MH_let:NwN \if_num:w =\ifnum}{}
\@ifundefined{if_dim:w}{\MH_let:NwN \if_dim:w =\ifdim}{}
\@ifundefined{if_case:w}{\MH_let:NwN \if_case:w =\ifcase}{}
\@ifundefined{or:}{\MH_let:NwN \or:=\or}{}
%    \end{macrocode}
%  \end{macro}
%  \end{macro}
%  \end{macro}
%  \end{macro}
%  \end{macro}
%  \end{macro}
%  \end{macro}
%  \end{macro}
%  \begin{macro}{\MH_cs_to_str:N}
%  Strip off the backslash of a macro name.
%    \begin{macrocode}
\def\MH_cs_to_str:N {\expandafter\@gobble\string}
%    \end{macrocode}
%  \end{macro}
%
%  \begin{macro}{\MH_protected:}
%  \begin{macro}{\MH_setlength:dn}
%  \changes{v1.0a}{2004/08/23}{Fixed bug in \cs{MH_setlength:dn} so that
%  it now uses the \pkg{calc} definition of \cs{setlength}}
%  \begin{macro}{\MH_addtolength:dn}
%  \changes{v1.0a}{2004/08/23}{Fixed bug in \cs{MH_addtolength:dn} so that
%  it now uses the \pkg{calc} definition of \cs{addtolength}}
%  We might as well make use of some of the extended features from
%  \eTeX. We use \cs{dimexpr} for some simple calculations as it
%  saves a lot of the scanning that goes on inside \pkg{calc}. The
%  \cs{protected} primitive comes in handy when we want to declare a
%  robust command, that cannot be `robustified' with
%  \cs{DeclareRobustCommand}. If we don't have \eTeX{} we'll just let
%  our private macros be aliases for the less effective alternatives.
%    \begin{macrocode}
\@ifundefined{eTeXversion}
  {
    \MH_let:NwN \MH_protected:\relax
    \def\MH_setlength:dn{\setlength}
    \def\MH_addtolength:dn{\addtolength}
  }
  {
    \MH_let:NwN \MH_protected:\protected
    \def\MH_setlength:dn #1#2{#1=\dimexpr#2\relax\relax}
    \def\MH_addtolength:dn #1#2{\advance#1 \dimexpr#2\relax\relax}
  }
%    \end{macrocode}
%  \end{macro}
%  \end{macro}
%  \end{macro}

%  \begin{macro}{\MH_keyval_alias_with_addon:nnnn}
%  \begin{macro}{\MH_keyval_alias:nnn}
%  A way to make aliases with \pkg{keyval}. This will come in
%  handy later.
%    \begin{macrocode}
\def\MH_keyval_alias_with_addon:nnnn #1#2#3#4{
  \@namedef{KV@#1@#2}{\@nameuse{KV@#1@#3}#4}
  \@namedef{KV@#1@#2@default}{\@nameuse{KV@#1@#3@default}#4}}
\def\MH_keyval_alias:nnn #1#2#3{
  \MH_keyval_alias_with_addon:nnnn {#1}{#2}{#3}{}}
%    \end{macrocode}
%  \end{macro}
%  \end{macro}
%  \begin{macro}{\MH_use_choice_i:nnnn}
%  \begin{macro}{\MH_use_choice_ii:nnnn}
%  \begin{macro}{\MH_use_choice_iii:nnnn}
%  \begin{macro}{\MH_use_choice_iv:nnnn}
%  I need to be able to pick up individual arguments in a list of
%  four (similar to \cs{@secondoftwo}).
%    \begin{macrocode}
\def\MH_use_choice_i:nnnn #1#2#3#4{#1}
\def\MH_use_choice_ii:nnnn #1#2#3#4{#2}
\def\MH_use_choice_iii:nnnn #1#2#3#4{#3}
\def\MH_use_choice_iv:nnnn #1#2#3#4{#4}
%    \end{macrocode}
%  \end{macro}
%  \end{macro}
%  \end{macro}
%  \end{macro}
%  \begin{macro}{\MH_nospace_ifnextchar:Nnn}
%  \begin{macro}{\MH_nospace_nextchar:}
%  \begin{macro}{\MH_nospace_testopt:nn}
%  \begin{macro}{\MH_nospace_protected_testopt:n}
%  Scanning for the next character but disallow spaces.
%    \begin{macrocode}
\long\def\MH_nospace_ifnextchar:Nnn #1#2#3{
  \MH_let:NwN\reserved@d=~#1
  \def\reserved@a{#2}
  \def\reserved@b{#3}
  \futurelet\@let@token\MH_nospace_nextchar:
}
\def\MH_nospace_nextchar:{
  \if_meaning:NN \@let@token\reserved@d
    \MH_let:NwN \reserved@b\reserved@a
  \fi:
  \reserved@b
}
\long\def\MH_nospace_testopt:nn #1#2{
  \MH_nospace_ifnextchar:Nnn[
    {#1}
    {#1[{#2}]}
}
\def\MH_nospace_protected_testopt:n #1{
  \if_meaning:NN \protect\@typeset@protect
    \expandafter\MH_nospace_testopt:nn
  \else:
    \@x@protect#1
  \fi:
}
%    \end{macrocode}
%  \end{macro}
%  \end{macro}
%  \end{macro}
%  \end{macro}
%  \begin{macro}{\kernel@ifnextchar}
%  \begin{macro}{\MH_kernel_xargdef:nwwn}
%  \begin{macro}{\MH_nospace_xargdef:nwwn}
%  \begin{macro}{\MHPrecedingSpacesOff}
%  \begin{macro}{\MHPrecedingSpacesOn}
%  The code for the space sensitive peek ahead.
%    \begin{macrocode}
\@ifundefined{kernel@ifnextchar}
  {\MH_let:NwN \kernel@ifnextchar \@ifnextchar}
  {}
\MH_let:NwN \MH_kernel_xargdef:nwwn \@xargdef
\long\def\MH_nospace_xargdef:nwwn #1[#2][#3]#4{
  \@ifdefinable#1{
     \expandafter\def\expandafter#1\expandafter{
          \expandafter
          \MH_nospace_protected_testopt:n
          \expandafter
          #1
          \csname\string#1\endcsname
          {#3}}
       \expandafter\@yargdef
          \csname\string#1\endcsname
           \tw@
           {#2}
           {#4}}}
\providecommand*\MHPrecedingSpacesOff{
  \MH_let:NwN \@xargdef \MH_nospace_xargdef:nwwn
}
\providecommand*\MHPrecedingSpacesOn{
  \MH_let:NwN \@xargdef \MH_kernel_xargdef:nwwn
}
%    \end{macrocode}
%  \end{macro}
%  \end{macro}
%  \end{macro}
%  \end{macro}
%  \end{macro}
%
%  \begin{macro}{\MH_group_align_safe_begin:}
%  \begin{macro}{\MH_group_align_safe_end:}
%    \begin{macrocode}
\def \MH_group_align_safe_begin: {\iffalse{\fi\ifnum0=`}\fi}
\def \MH_group_align_safe_end:   {\ifnum0=`{}\fi}
%    \end{macrocode}
%  \end{macro}
%  \end{macro}
%
%    \begin{macrocode}
%</package>
%    \end{macrocode}
%
%  \Finale
\endinput

%        (quote the arguments according to the demands of your shell)
%
% Documentation:
%    (a) If mhsetup.drv is present:
%           latex mhsetup.drv
%    (b) Without mhsetup.drv:
%           latex mhsetup.dtx; ...
%    The class ltxdoc loads the configuration file ltxdoc.cfg
%    if available. Here you can specify further options, e.g.
%    use A4 as paper format:
%       \PassOptionsToClass{a4paper}{article}
%
%    Programm calls to get the documentation (example):
%       pdflatex mhsetup.dtx
%       makeindex -s gind.ist mhsetup.idx
%       pdflatex mhsetup.dtx
%       makeindex -s gind.ist mhsetup.idx
%       pdflatex mhsetup.dtx
%
% Installation:
%    TDS:tex/latex/mh/mhsetup.sty
%    TDS:doc/latex/mh/mhsetup.pdf
%    TDS:source/latex/mh/mhsetup.dtx
%
%<*ignore>
\begingroup
  \def\x{LaTeX2e}
\expandafter\endgroup
\ifcase 0\ifx\install y1\fi\expandafter
         \ifx\csname processbatchFile\endcsname\relax\else1\fi
         \ifx\fmtname\x\else 1\fi\relax
\else\csname fi\endcsname
%</ignore>
%<*install>
\input docstrip.tex
\Msg{************************************************************************}
\Msg{* Installation}
\Msg{* Package: mhsetup 2007/12/03 v1.2 mhsetup (MH)}
\Msg{************************************************************************}

\keepsilent
\askforoverwritefalse

\preamble

This is a generated file.

Copyright (C) 2002-2007 by Morten Hoegholm <mh.ctan@gmail.com>

This work may be distributed and/or modified under the
conditions of the LaTeX Project Public License, either
version 1.3 of this license or (at your option) any later
version. The latest version of this license is in
   http://www.latex-project.org/lppl.txt
and version 1.3 or later is part of all distributions of
LaTeX version 2005/12/01 or later.

This work has the LPPL maintenance status "maintained".

This Current Maintainer of this work is Morten Hoegholm.

This work consists of the main source file mhsetup.dtx
and the derived files
   mhsetup.sty, mhsetup.pdf, mhsetup.ins, mhsetup.drv.

\endpreamble

\generate{%
  \file{mhsetup.ins}{\from{mhsetup.dtx}{install}}%
  \file{mhsetup.drv}{\from{mhsetup.dtx}{driver}}%
  \usedir{tex/latex/mh}%
  \file{mhsetup.sty}{\from{mhsetup.dtx}{package}}%
}

\obeyspaces
\Msg{************************************************************************}
\Msg{*}
\Msg{* To finish the installation you have to move the following}
\Msg{* file into a directory searched by TeX:}
\Msg{*}
\Msg{*     mhsetup.sty}
\Msg{*}
\Msg{* To produce the documentation run the file `mhsetup.drv'}
\Msg{* through LaTeX.}
\Msg{*}
\Msg{* Happy TeXing!}
\Msg{*}
\Msg{************************************************************************}

\endbatchfile
%</install>
%<*ignore>
\fi
%</ignore>
%<*driver>
\NeedsTeXFormat{LaTeX2e}
\ProvidesFile{mhsetup.drv}%
  [2007/12/03 v1.2 programming setup (MH)]
\documentclass{ltxdoc}
\usepackage{mhsetup}

\addtolength\oddsidemargin{30pt}
\providecommand*\pkg[1]{\textsf{#1}}
\providecommand*\env[1]{\texttt{#1}}
\providecommand*\opt[1]{\texttt{#1}}
\def\MakePrivateLetters{%
  \makeatletter\catcode`\_=11\relax \catcode`\:=11\relax
}



\def\MHmarg#1{{\ttfamily\char`\{}\meta{#1}{\ttfamily\char`\}}\endgroup}
\DeclareRobustCommand\marg{\begingroup\MakePrivateLetters\MHmarg}%
\def\MHarg#1{{\ttfamily\char`\{#1\ttfamily\char`\}}\endgroup}
\DeclareRobustCommand\arg{\begingroup\MakePrivateLetters\MHarg}%
\def\MHcs#1{\texttt{\char`\\#1}\endgroup}
\DeclareRobustCommand\cs{\begingroup\MakePrivateLetters\MHcs}
\makeatletter
\providecommand*\eTeX{$\m@th\varepsilon$-\TeX}
\makeatother
\begin{document}
  \DocInput{mhsetup.dtx}
\end{document}
%</driver>
%  \fi
%
%  \CheckSum{297}
%
%  \GetFileInfo{mhsetup.drv}
%
%  \title{The \pkg{mhsetup} package\thanks{This package has version
%  number \fileversion, last revised on \filedate.}}
%  \author{Morten H\o gholm}
%  \date{\filedate}
%  \maketitle
%
%  \begin{abstract}
%    The \pkg{mhsetup} package provides tools for a \LaTeX{}
%    programming environment similar to the one described in
%    \texttt{expl3} on \textsc{ctan} although not as extensive. It is
%    a required part of both the \pkg{mathtools} and \pkg{empheq}
%    packages.
%  \end{abstract}
%
%  The description below was made before the extensive changes made to
%  the expl3 code available from the LaTeX Project website.
%
%  \section{The new internal syntax}
%
%  
%  The \LaTeX3 package \pkg{ldcsetup} defines the command
%  \cs{InternalSyntaxOn} which makes |_| and |:| letters and then
%  automatically restores the category codes at the end of the
%  package. This usually works fine but when you try to load
%  \pkg{amstext} you will experience that \TeX{} goes into an
%  infinite loop. Packages containing code like
%  |\@for\@tempa:=\@tempb\do{...}| will not work correctly either,
%  thus we provide an alternative version here with the pair of
%  commands
%  \DescribeMacro{\MHInternalSyntaxOn}\cs{MHInternalSyntaxOn} and
%  \DescribeMacro{\MHInternalSyntaxOff}\cs{MHInternalSyntaxOff}. They
%  are to be used only as a pair, because \cs{MHInternalSyntaxOn}
%  defines \cs{MHInternalSyntaxOff} so that it restores the category
%  codes correctly.
%
%
%  \section{Handling optional arguments}
%
%  The standard behavior of scanning for optional arguments in
%  \LaTeX{} allows any number of spaces preceding the optional
%  argument and that is not always good in math. For that reason
%  \pkg{amsmath} makes sure that commands like |\\| disallows spaces
%  before the optional argument but at the same time it fails to
%  provide ``safe'' environments. What would you expect from the
%  following input?
%  \begin{verbatim}
%  \[
%    \begin{gathered}
%      [v] = 100 \\
%      [t] = 200
%    \end{gathered}
%  \]
%  \end{verbatim}
%  \LaTeX{} will see the \texttt{[v]} as an optional argument of
%  \env{gathered} and use it. In this case the test inside
%  \env{gathered} checks if it's a \texttt{t} or \texttt{b} and if
%  it's neither it'll choose \cs{vcenter} internally. So you get no
%  warning, only missing output. Another example, this time from the
%  \pkg{empheq} package used with its \opt{overload} option: If
%  preceding spaces are allowed, the input
%  \begin{verbatim}
%    \begin{gather}
%      [a] = [b]
%    \end{gather}
%  \end{verbatim}
%  results in the rather strange error message
%  \begin{verbatim}
%    ! Package keyval Error: a undefined.
%  \end{verbatim}
%
%  When using \cs{newcommand} etc.\ for defining commands and
%  environments with optional arguments, the peek ahead is done by
%  \cs{kernel@ifnextchar} (since \LaTeX{} release 2003/12/01, else
%  \cs{@ifnextchar}) and it is \emph{hardwired at definition time} by
%  \cs{@xargdef}. With the commands
%  \DescribeMacro{\MHPrecedingSpacesOff}\cs{MHPrecedingSpacesOff} and
%  \DescribeMacro{\MHPrecedingSpacesOn}\cs{MHPrecedingSpacesOn}
%  \pkg{mhsetup} provides an interface to define commands and
%  environments where the optional argument cannot have preceding
%  spaces. You simply wrap them around the definitions:
%  \begin{verbatim}
%  \MHPrecedingSpacesOff
%  \newenvironment*{test}[1][default]{Start, arg: (#1)}{Ending.}
%  \MHPrecedingSpacesOn
%  \begin{test}
%    [text]
%  \end{test}
%  \begin{test}[text]
%  \end{test}
%  \end{verbatim}
%  \MHPrecedingSpacesOff
%  \newenvironment*{test}[1][default]{Start, arg: (#1)}{Ending.}
%  \MHPrecedingSpacesOn
%  \begin{quote}
%  \begin{test}
%    [text]
%  \end{test}
%  \begin{test}[text]
%  \end{test}
%  \end{quote}
%  It is of somewhat limited use in commands (control words in \TeX{}
%  terminology), because \TeX{} discards the spaces. The exception is
%  \emph{control symbols} where \TeX{} obeys following spaces but
%  there are rather few of them available. All is not lost however.
%  In the \env{aligned} environment from \pkg{amsmath} (shown below)
%  a command is used as argument grabber.
%  \begin{verbatim}
%  \newenvironment{aligned}{%
%    \let\@testopt\alignsafe@testopt
%    \aligned@a
%  }{%
%    \crcr\egroup
%    \restorecolumn@
%    \egroup
%  }
%  \newcommand{\aligned@a}[1][c]{\start@aligned{#1}\m@ne}
%  \end{verbatim}
%  By applying our trick on the grabber function, we get a space
%  obeying version:
%  \begin{verbatim}
%  \MHPrecedingSpacesOff
%  \renewcommand*\aligned@a[1][c]{\start@aligned{#1}\m@ne}
%  \MHPrecedingSpacesOn
%  \end{verbatim}
%  This way a nested \env{aligned} environment is still safe from
%  empty first cells.
%
%  \StopEventually{}
%
%
%  \section{First bits of a new programming environment}
%
%
%    \begin{macrocode}
%<*package>
\ProvidesPackage{mhsetup}%
  [2007/12/03 v1.2 programming setup (MH)]
%    \end{macrocode}
%  \subsection{The new internal syntax}
%  \begin{macro}{\MHInternalSyntaxOn}
%  \begin{macro}{\MHInternalSyntaxOff}
%  Almost copy of \cs{InternalSyntaxOn}.
%    \begin{macrocode}
\def\MHInternalSyntaxOn{
  \edef\MHInternalSyntaxOff{%
  \catcode`\noexpand\~=\the\catcode`\~\relax
  \catcode`\noexpand\ =\the\catcode`\ \relax
  \catcode`\noexpand\^^I=\the\catcode`\^^I\relax
  \catcode`\noexpand\@=\the\catcode`\@\relax
  \catcode`\noexpand\:=\the\catcode`\:\relax
  \catcode`\noexpand\_=\the\catcode`\_\relax
  \endlinechar=\the\endlinechar\relax
  }%
  \catcode`\~=10\relax
  \catcode`\ =9\relax
  \catcode`\^^I=9\relax
  \makeatletter
  \catcode`\_=11\relax
  \catcode`\:=11\relax
  \endlinechar=` %
  \relax
}
\MHInternalSyntaxOn
\AtEndOfPackage{\MHInternalSyntaxOff}
%    \end{macrocode}
%  \end{macro}
%  \end{macro}
%  \subsection{Programming tools}
%
%  The whole idea is to provide programming tools that are convenient
%  but not yet widely available. I hope this'll be obsolete soon!
%
%  Firstly we setup a few helper functions.
%  \begin{macro}{\MH_let:NwN}
%  An alias for \cs{let}.
%    \begin{macrocode}
\let\MH_let:NwN \let
%    \end{macrocode}
%  \end{macro}
%  \begin{macro}{\MH_let:cN}
%  This one takes a \cs{csname}-\cs{endcsname} name and \cs{let}s it
%  to a single macro. We'll use this to setup our conditionals.
%    \begin{macrocode}
\def\MH_let:cN #1#2{
  \expandafter\MH_let:NwN \csname#1\endcsname#2}
%    \end{macrocode}
%  \end{macro}
%  \begin{macro}{\MH_let:cc}
%  This one has takes a \cs{csname}-\cs{endcsname} name and \cs{let}s
%  it to a another \cs{csname}-\cs{endcsname} name. To be used in
%  constructions with weird characters like \texttt{*} or alike in
%  them and can take a \cs{global} prefix if wanted (we want that
%  later on).
%    \begin{macrocode}
\def\MH_let:cc #1#2{
  \expandafter\MH_let:NwN\csname#1\expandafter\endcsname
  \csname#2\endcsname}
%    \end{macrocode}
%  \end{macro}
%
%  \begin{macro}{\MH_new_boolean:n}
%  \begin{macro}{\MH_set_boolean_F:n}
%  \begin{macro}{\MH_set_boolean_T:n}
%  \begin{macro}{\MH_if_boolean:nTF}
%  \begin{macro}{\MH_if_boolean:nT}
%  \begin{macro}{\MH_if_boolean:nF}
%  Sets up conditionals. For instance
%  \begin{quote}
%    \cs{MH_new_boolean:n} \marg{name}
%  \end{quote}
%  defines the boolean \meta{name} but also the conditional
%  \cs{if_boolean_\meta{name}:} to be used in the ordinary
%  \begin{tabbing}
%    \mbox{}\quad\= \kill
%    \cs{if_boolean_\meta{name}:}\\
%    \> \meta{true code} \\
%    \cs{else:} \\
%    \> \meta{false code} \\
%    \cs{fi:}
%  \end{tabbing}
%  There is also a more ``\LaTeX-like'' interface available by using
%  the commands
%  \begin{quote}
%    \cs{MH_if_boolean:nT}\marg{name}\marg{arg}
%  \end{quote}
%  which will execute the argument if the current value of the
%  boolean is `true' while
%  \begin{quote}
%  \cs{MH_if_boolean:nF}\marg{name}\marg{arg}
%  \end{quote}
%  is the equivalent with `false'. Finally we have
%  \begin{quote}
%  \cs{MH_if_boolean:nTF}\marg{name}\marg{true code}\marg{false code}.
%  \end{quote}
%  This is the interface I have used in this package.
%
%  Initially \cs{if_boolean_\meta{name}:} is `false'. This can be
%  changed by saying
%  \begin{tabbing}
%    \quad \= \LaTeX:\quad \= \kill
%    \> \TeX: \>   \cs{boolean_\meta{name}_true:} \quad \emph{or}\\
%    \> \LaTeX: \> \cs{MH_set_boolean_T:n}\marg{name}
%  \end{tabbing}
%  and changed back again by
%  \begin{tabbing}
%    \quad \= \LaTeX:\quad \= \kill
%    \> \TeX: \>   \cs{boolean_\meta{name}_false:} \quad \emph{or}\\
%    \> \LaTeX: \> \cs{MH_set_boolean_F:n}\marg{name}
%  \end{tabbing}
%
%  And yes, we're also using alternative names for \cs{else} and
%  \cs{fi} now. That way a simple search and replace will be all that
%  is needed for this package to be a certified \LaTeX3 package (well,
%  maybe a little more is needed, but not much).
%    \begin{macrocode}
\def\MH_new_boolean:n #1{
  \expandafter\@ifdefinable\csname if_boolean_#1:\endcsname{
    \@namedef{boolean_#1_true:}
      {\MH_let:cN{if_boolean_#1:}\iftrue}
    \@namedef{boolean_#1_false:}
      {\MH_let:cN{if_boolean_#1:}\iffalse}
    \@nameuse{boolean_#1_false:}%
  }
}
\def\MH_set_boolean_F:n #1{ \@nameuse{boolean_#1_false:} }
\def\MH_set_boolean_T:n #1{ \@nameuse{boolean_#1_true:} }
\def\MH_if_boolean:nTF #1{
  \@nameuse{if_boolean_#1:}
    \expandafter\@firstoftwo
  \else:
    \expandafter\@secondoftwo
  \fi:
}
\def\MH_if_boolean:nT #1{
  \@nameuse{if_boolean_#1:}
    \expandafter\@firstofone
  \else:
    \expandafter\@gobble
  \fi:
}
\def\MH_if_boolean:nF #1{
  \@nameuse{if_boolean_#1:}
    \expandafter\@gobble
  \else:
    \expandafter\@firstofone
  \fi:
}
%    \end{macrocode}
%  \end{macro}
%  \end{macro}
%  \end{macro}
%  \end{macro}
%  \end{macro}
%  \end{macro}
%
%  \begin{macro}{\if:w}
%  \changes{v1.0b}{2004/10/10}{Use different test}
%  \begin{macro}{\if_meaning:NN}
%  \changes{v1.0b}{2004/10/10}{Use different test}
%  \begin{macro}{\else:}
%  \changes{v1.0b}{2004/10/10}{Use different test}
%  \begin{macro}{\fi:}
%  \changes{v1.0b}{2004/10/10}{Use different test}
%  \begin{macro}{\if_num:w}
%  \changes{v1.0b}{2004/10/10}{Use different test}
%  \begin{macro}{\if_dim:w}
%  \changes{v1.0b}{2004/10/10}{Use different test}
%  \begin{macro}{\if_case:w}
%  \changes{v1.0b}{2004/10/10}{Use different test}
%  \begin{macro}{\or:}
%  \changes{v1.0b}{2004/10/10}{Use different test}
%  Copies of \TeX{} primitives.
%    \begin{macrocode}
\@ifundefined{if:w}{\MH_let:NwN \if:w =\if}{}
\@ifundefined{if_meaning:NN}{\MH_let:NwN \if_meaning:NN =\ifx}{}
\@ifundefined{else:}{\MH_let:NwN \else:=\else}{}
\@ifundefined{fi:}{\MH_let:NwN \fi:=\fi}{}
\@ifundefined{if_num:w}{\MH_let:NwN \if_num:w =\ifnum}{}
\@ifundefined{if_dim:w}{\MH_let:NwN \if_dim:w =\ifdim}{}
\@ifundefined{if_case:w}{\MH_let:NwN \if_case:w =\ifcase}{}
\@ifundefined{or:}{\MH_let:NwN \or:=\or}{}
%    \end{macrocode}
%  \end{macro}
%  \end{macro}
%  \end{macro}
%  \end{macro}
%  \end{macro}
%  \end{macro}
%  \end{macro}
%  \end{macro}
%  \begin{macro}{\MH_cs_to_str:N}
%  Strip off the backslash of a macro name.
%    \begin{macrocode}
\def\MH_cs_to_str:N {\expandafter\@gobble\string}
%    \end{macrocode}
%  \end{macro}
%
%  \begin{macro}{\MH_protected:}
%  \begin{macro}{\MH_setlength:dn}
%  \changes{v1.0a}{2004/08/23}{Fixed bug in \cs{MH_setlength:dn} so that
%  it now uses the \pkg{calc} definition of \cs{setlength}}
%  \begin{macro}{\MH_addtolength:dn}
%  \changes{v1.0a}{2004/08/23}{Fixed bug in \cs{MH_addtolength:dn} so that
%  it now uses the \pkg{calc} definition of \cs{addtolength}}
%  We might as well make use of some of the extended features from
%  \eTeX. We use \cs{dimexpr} for some simple calculations as it
%  saves a lot of the scanning that goes on inside \pkg{calc}. The
%  \cs{protected} primitive comes in handy when we want to declare a
%  robust command, that cannot be `robustified' with
%  \cs{DeclareRobustCommand}. If we don't have \eTeX{} we'll just let
%  our private macros be aliases for the less effective alternatives.
%    \begin{macrocode}
\@ifundefined{eTeXversion}
  {
    \MH_let:NwN \MH_protected:\relax
    \def\MH_setlength:dn{\setlength}
    \def\MH_addtolength:dn{\addtolength}
  }
  {
    \MH_let:NwN \MH_protected:\protected
    \def\MH_setlength:dn #1#2{#1=\dimexpr#2\relax\relax}
    \def\MH_addtolength:dn #1#2{\advance#1 \dimexpr#2\relax\relax}
  }
%    \end{macrocode}
%  \end{macro}
%  \end{macro}
%  \end{macro}

%  \begin{macro}{\MH_keyval_alias_with_addon:nnnn}
%  \begin{macro}{\MH_keyval_alias:nnn}
%  A way to make aliases with \pkg{keyval}. This will come in
%  handy later.
%    \begin{macrocode}
\def\MH_keyval_alias_with_addon:nnnn #1#2#3#4{
  \@namedef{KV@#1@#2}{\@nameuse{KV@#1@#3}#4}
  \@namedef{KV@#1@#2@default}{\@nameuse{KV@#1@#3@default}#4}}
\def\MH_keyval_alias:nnn #1#2#3{
  \MH_keyval_alias_with_addon:nnnn {#1}{#2}{#3}{}}
%    \end{macrocode}
%  \end{macro}
%  \end{macro}
%  \begin{macro}{\MH_use_choice_i:nnnn}
%  \begin{macro}{\MH_use_choice_ii:nnnn}
%  \begin{macro}{\MH_use_choice_iii:nnnn}
%  \begin{macro}{\MH_use_choice_iv:nnnn}
%  I need to be able to pick up individual arguments in a list of
%  four (similar to \cs{@secondoftwo}).
%    \begin{macrocode}
\def\MH_use_choice_i:nnnn #1#2#3#4{#1}
\def\MH_use_choice_ii:nnnn #1#2#3#4{#2}
\def\MH_use_choice_iii:nnnn #1#2#3#4{#3}
\def\MH_use_choice_iv:nnnn #1#2#3#4{#4}
%    \end{macrocode}
%  \end{macro}
%  \end{macro}
%  \end{macro}
%  \end{macro}
%  \begin{macro}{\MH_nospace_ifnextchar:Nnn}
%  \begin{macro}{\MH_nospace_nextchar:}
%  \begin{macro}{\MH_nospace_testopt:nn}
%  \begin{macro}{\MH_nospace_protected_testopt:n}
%  Scanning for the next character but disallow spaces.
%    \begin{macrocode}
\long\def\MH_nospace_ifnextchar:Nnn #1#2#3{
  \MH_let:NwN\reserved@d=~#1
  \def\reserved@a{#2}
  \def\reserved@b{#3}
  \futurelet\@let@token\MH_nospace_nextchar:
}
\def\MH_nospace_nextchar:{
  \if_meaning:NN \@let@token\reserved@d
    \MH_let:NwN \reserved@b\reserved@a
  \fi:
  \reserved@b
}
\long\def\MH_nospace_testopt:nn #1#2{
  \MH_nospace_ifnextchar:Nnn[
    {#1}
    {#1[{#2}]}
}
\def\MH_nospace_protected_testopt:n #1{
  \if_meaning:NN \protect\@typeset@protect
    \expandafter\MH_nospace_testopt:nn
  \else:
    \@x@protect#1
  \fi:
}
%    \end{macrocode}
%  \end{macro}
%  \end{macro}
%  \end{macro}
%  \end{macro}
%  \begin{macro}{\kernel@ifnextchar}
%  \begin{macro}{\MH_kernel_xargdef:nwwn}
%  \begin{macro}{\MH_nospace_xargdef:nwwn}
%  \begin{macro}{\MHPrecedingSpacesOff}
%  \begin{macro}{\MHPrecedingSpacesOn}
%  The code for the space sensitive peek ahead.
%    \begin{macrocode}
\@ifundefined{kernel@ifnextchar}
  {\MH_let:NwN \kernel@ifnextchar \@ifnextchar}
  {}
\MH_let:NwN \MH_kernel_xargdef:nwwn \@xargdef
\long\def\MH_nospace_xargdef:nwwn #1[#2][#3]#4{
  \@ifdefinable#1{
     \expandafter\def\expandafter#1\expandafter{
          \expandafter
          \MH_nospace_protected_testopt:n
          \expandafter
          #1
          \csname\string#1\endcsname
          {#3}}
       \expandafter\@yargdef
          \csname\string#1\endcsname
           \tw@
           {#2}
           {#4}}}
\providecommand*\MHPrecedingSpacesOff{
  \MH_let:NwN \@xargdef \MH_nospace_xargdef:nwwn
}
\providecommand*\MHPrecedingSpacesOn{
  \MH_let:NwN \@xargdef \MH_kernel_xargdef:nwwn
}
%    \end{macrocode}
%  \end{macro}
%  \end{macro}
%  \end{macro}
%  \end{macro}
%  \end{macro}
%
%  \begin{macro}{\MH_group_align_safe_begin:}
%  \begin{macro}{\MH_group_align_safe_end:}
%    \begin{macrocode}
\def \MH_group_align_safe_begin: {\iffalse{\fi\ifnum0=`}\fi}
\def \MH_group_align_safe_end:   {\ifnum0=`{}\fi}
%    \end{macrocode}
%  \end{macro}
%  \end{macro}
%
%    \begin{macrocode}
%</package>
%    \end{macrocode}
%
%  \Finale
\endinput
